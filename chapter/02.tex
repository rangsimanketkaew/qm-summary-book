% LaTeX source for ``Summary of Quantum Chemistry''
% Copyright (c) 2023 รังสิมันต์ เกษแก้ว (Rangsiman Ketkaew).

\chapter{Hartree-Fock}

\section{เกริ่นนำ}

เรามีนิยามทางคณิตศาสตร์ของ Hamiltonian ของโมเลกุล (แบบ non-relativistic) สำหรับระบบหลายอิเล็กตรอนก็คือ

\begin{equation}
    \begin{aligned}
        H = &
        \left[ \right.
        -\sum_{i} \frac{\hbar^{2}}{2m_{e}} \nabla^{2}_{i}
        -\sum_{i,I} \frac{Z_{I} e^{2}}{4\pi\epsilon_{0} |r_{i} - R_{I}|} \\
            & +\sum_{j<i} \frac{e^{2}}{4\pi\epsilon r_{ij}}
        +\sum_{I>J} \frac{Z_{I} Z_{J} e^{2}}{4\pi\epsilon_{9} |R_{I} - R_{J}|}
        \left. \right]
    \end{aligned}
\end{equation}

สำหรับสมการข้างต้นนี้เราจำเป็นที่จะต้องรู้ Atomic type แล้วก็พิกัดหรือตำแหน่งของอะตอม (นิวเคลียส)
โดย Position นั้นไม่จำเป็นที่ต้องรู้แบบเป๊ะ ๆ ก็ได้แต่ว่าควรจะต้องเป็นพิกัดที่สมเหตุสมผลเพราะว่าเราสามารถทำการปรับ
geometry ของโมเลกุลทีหลังได้ แล้วก็ Exact Wavefunction ของระบบหลายอิเล็กตรอนนั้นไม่สามารถแก้ได้
ดังนั้นเราจึงจำเป็นที่จะต้องใช้การประมาณหรือ Approximations หลาย ๆ อันเข้ามาช่วย นอกจากนี้เรายังใช้หลักการผันแปร
(Variational Principle) ด้วยเพื่อช่วยในการหา trial Wavefunction ที่ดีที่สุดจาก Function class

Variational Principle ก็คือ ยิ่ง trial Function นั้นเข้าใกล้ true Wavefunction มากเท่าไหร่
พลังงานรวมของระบบหรือ Energy expectation value เราจะยิ่งมีค่าต่ำลงเท่านั้น หรือถ้าตีความง่าย ๆ
ก็คือพลังงานของ trial Wavefunction นั้นจะไม่มีทางที่จะต่ำไปว่าพลังงานของ true Wavefunction ได้

สำหรับการพิสูจน์ของ Variational Principle นั้นสามารถหาอ่านรายละเอียดได้ตามหนังสือ Electronic
structure

สรุปหลักการ Variational ก็คือ ยิ่งพลังงานต่ำยิ่งดีเพราะว่าเรายิ่งได้ Wavefunction ที่ถูถต้องมากขึ้น

\section{สมการ Hartree=Fock}

เราสามารถเขียน Wavefunction ได้ในรูปฟังก์ชันของผลคูณแบบ anti-symmetric ได้นั่นก็คือ Slater Determinant
โดยองค์ประกอบสำคัญของ Slater Determinant ก็คื Atomic Orbital ซึ่งในเบื้องต้นนั้นเราจะทำการ ignore
สปินออกไปก่อนเพื่อความง่ายในการอธิบาย โดยเราจะทำการ Assume ว่าสถานะหรือ State ทั้งหมดนั้นจะเป็นแบบ
Doubly Occupied ก็คือจำนวนอิเล็กตรอนทั้งหมดของระบบนั้นเท่ากับ $2N!$ แล้วก็ Atomic Orbital
นั้นเป็น Orthonormal (Orthogonal + Normality)

ในการปฏิบัตินั้นการแก้สมการ Hartree-Fock (HF Equation) ยังไม่สามารถแก้ออกมาได้แบบถูกต้อง 100\%
เนื่องจากว่าการคำนวณที่เกี่ยวข้องกับ Wavefunction นั้นยังต้องมีการใช้ Numerical Representation เข้ามาช่วย

จริง ๆ แล้ว Atomic Orbitals นั้นก็คือ Wavefunction นั่นเอง โดยเราสามารถเขียน Wavefunction
ได้โดยใช้ฟังก์ชันทางคณิตศาสตร์ที่เรียกว่า Basis Function ซึ่งจริง ๆ แล้วเราจะใช้ฟังก์ชันอะไรก็ได้มาเป็น
Basis Function แต่ฟังก์ชันที่ได้รับความนิยมและมีความสมเหตุสมผลนั้นก็คือฟังก์ชัน Gaussian
เพราะว่าเป็นฟังก์ชันทางคณิตศาสตร์ที่มีคุณสมบัติหลาย ๆ อย่างตรงกับพฤติกรรม (behavior) ของ Wavefunction
แล้วก็มีพารามิเตอร์ที่สามารถปรับค่าได้

โดยปกติแล้ว Basis Function นั้นจะถูกกำหนดให้มีจุดศูนย์กลางอยู่ที่อะตอม (Centered on Atom)
ทีนี้เราก็ทำการยัด Atomic Orbital (Wavefunction) ที่เราใช้ Basis Function เข้าไปใน Slater
Determinant แล้วก็เอาใส่เข้าไปใน Variation Equation (หรือสมการ Schr\"{o}dinger)
แล้วก็เราใช้คณิตศาสตร์นิดหน่อยในการจัดรูปสมการแล้วเราก็จะได้สมการ Roothaan-Hall (R-H) ดังนี้

\begin{equation}
    FC = \epsilon SC
\end{equation}

โดยที่ $C$ คือเมทริกซ์ที่เก็บค่าสัมประสิทธิ์ของ Molecular Orbitals เอาไว้ แล้วก็ $F$ คือ Fock Matrix
ซึ่งก็จะเป็นตัวแทนของ Hamiltonian ส่วน $S$ ก็คือ Overlap Matrix

สมการ R-H นั้นมีความซับซ้อนระดับนึงจึงทำให้การแก้สมการนั้นทำได้ยาก แต่ว่านักวิทยาศาสตร์ก็หาวิธีจะได้
ประเด็นก็คือการแก้สมการ R-H นั้นมันมีตัวที่เป็นปัญหาก็คือ Fock Matrix ซึ่งจะมีการคำนวณ One-Electron
และ two-Electron Integral เข้ามาเกี่ยวข้องซึ่งเป็นการ Integral ของฟังก์ชัน 4 อันคูณกัน
(มีทั้งหมด 6 มิติ) ดังนั้นถ้าหากเรามีจำนวน Basis Function ทั้งหมด $N$ ฟังก์ชัน
การคำนวณนั้นจะมีความซับซ้อนถึง $M^{4}$ เลยทีเดียว ซึ่งเราไม่สามารถแบบตรงไปตรงมาได้

เริ่มต้นเลยคือเราจะทำการกำหนด initial Guess หรือ $C$ Matrix เริ่มต้นขึ้นมาก่อนแล้วก็ทำการแก้หา
Fock Matrix โดยใช้ Guess Matrix อันนี้ หลังจากนั้นจึงเป็นการแก้สมการ R-H แล้วก็หา $C$ Matrix
อันใหม่ออกมา แล้วก็ทำแบบนี้วนไปเรื่อย ๆ เราเรียกวิธีการแบบนี้ว่า Self-Consistent Loop

\section{Basis Functions}

ปัญหาถัดไปก็คือ Basis Function!! จริง ๆ แล้วมีวิธีหลาย ๆ อันมาก ๆ ที่เราสามารถใช้ในการสร้างหรือกำหนด
Basis Function ขึ้นมาได้แต่ว่าวิธีที่ง่ายที่สุดนั้นก็คือใช้ Slater Function เป็น Basis Function

HF Theory และ Basis Function นั้นเป็นตัวที่ทำให้เกิด Limit ของ Accuracy ของการคำนวณ ถ้า Basis
Function ที่ใช้นั้นดีอยู่แล้วก็ไม่ทำให้เกิด Error ใด ๆ เลยนั้นจะกลายเป็นว่า Accuracy นั้นจะมี Limitation
ที่ถูกจำกัดด้วย HF Theory เท่านั้น สำหรับปัญหาที่เป็นที่ทราบกันดีนั้นก็คือปริมาณบางอย่างที่ได้จากการคำนวณด้วย HF
นั้นไม่ถูกต้องและคลาดเคลื่อนไปเยอะ เช่น พลังงานการยึดเหนี่ยว (Binding Energies) แต่ว่าก็มีปริมาณ 
เช่น ความยาวพันธะ (Bond Distance) ที่ HF ค่อนข้างให้ผลการคำนวณที่ถูกต้อง

สำหรับการคำนวณปริมาณอื่น ๆ ก็เช่น HF นั้นจะ overestimate ค่า Frequencies ประมาณ 10\% แต่เราก็สามารถใช้
Scaling Factor มาทำ Correlation ได้ อย่างไรก็ตามถ้าพูดกันแบบแฟร์ ๆ ก็คือ Vibrational Frequencies
เป็นปริมาณที่คำนวณได้ค่อนข้างยากเพราะว่าจะขึ้นอยู่กับผลต่างของระดับพลังงานที่เล็กมาก ๆ

\section{สรุป}

วิธี HF (แบบ Unrestricted) นั้นเป็นวิธีพื้นฐาน (Basis) ของ Quantum Chemical Methods ซึ่งวิธี HF
เองนั้นให้ผลการคำนวณที่ไม่ถูกต้องดังนั้นสำหรับการคำนวณทางเคมีควอนตัมที่ต้องการความแม่นยำนั้นเราจะใช้วิธี
Post-HF แทน
