% LaTeX source for ``Summary of Quantum Chemistry''
% Copyright (c) 2023 รังสิมันต์ เกษแก้ว (Rangsiman Ketkaew).

\chapter{พลังงานรวมของโมเลกุล}

\section{เกริ่นนำ}

สิ่งที่สำคัญที่สุดในวิชากลศาสตร์ควอนตัม (quantum mechanics) โดยเฉพาะอย่างยิ่งเคมีควอนตัมนั้นก็คือฟังก์ชันคลื่น 
(wavefunction) เพราะว่า wavefunction เป็นเสมือนตู้เซฟที่เก็บความลับต่าง ๆ ของอะตอมหรือโมเลกุลเอาไว้
ถ้าหากเราสามารถหากุญแจหรือรู้รหัสในการไขตู้เซฟได้เราก็จะสามารถทำความเข้าระบบต่าง ๆ ทางเคมีได้อย่างทะลุปรุโปร่ง 
ประเด็นคือเราไม่รู้หน้าตาของ wavefunction นี่สิ ในปัจจุบันสิ่งที่เราทำได้คือแค่หา approximation ของ wavefunction 
ได้เท่านั้น ดังนั้นวิชาโครงสร้างเชิงอิเล็กทรอนิกส์จึงถือว่าเป็นแขนงนึงของเคมีควอนตัมที่มุ่งเน้นไปที่การพัฒนาทฤษฎีต่าง ๆ 
ที่เกี่ยวข้องกับคุณสมบัติเชิงอิเล็กทรอนิกส์ของอะตอมหรือโมเลกุลที่สอดคล้องกับพฤติกรรมของอิเล็กตรอนของระบบนั้น ๆ 
รวมไปถึงการพัฒนาวิธีจำลองทางคอมพิวเตอร์แบบต่าง ๆ เพื่อนำมาใช้เป็นเครื่องมือในการนำทฤษฎีต่าง ๆ ที่ถูกพัฒนาขึ้นมานั้น 
ไปใช้ในการคำนวณและศึกษาคุณสมบัติของระบบทางเคมีต่าง ๆ ต่อไป

\section{Total energy}

ก่อนที่จะเข้าสู่เนื้อหาที่เราจะต้องไปเจอกับฟังก์ชันคลื่นหรือ Wavefunction และสมการชโรดิงเงอร์หรือ Schrodinger 
equation นั้นผมอยากให้มาทำความเข้าใจกับพลังงานรวมของอะตอมกันก่อน เริ่มต้นก็คือเราจะทำการ assume 
ว่าอิเล็กตรอนนั้นจะมีสภาวะปกติที่สภาวะพื้น (ground state) ซึ่งพลังงานรวมของมันนั้นจะขึ้นอยู่กับ atomic coordinates

\begin{equation}
    E^{el}_{tot}(R_{1}, \dots, R_{N})
\end{equation}

โดยเราจะเริ่มด้วยนำการประมาณแบบแรกเข้ามาใช้เพื่อทำให้การแก้สมการของเรานั้นทำได้ง่ายขึ้นซึ่งวิธีการประมาณนั้นก็คือ 
Born-Oppenheimer approximation นั่นเองซึ่งเป็นการ assume ว่า \enquote{สำหรับมุมมองของอิเล็กตรอนนั้น 
นิวเคลียสก็เป็นเพียงแค่จุด ๆ หนึ่งเท่านั้นเอง ดังนั้นเราจึงสามารถแยก wavefunction ของอิเล็กตรอนออกมาจาก 
wavefunction ของนิวเคลียสได้} ซึ่ง approximation อันนี้มีประโยชน์มาก ๆ เพราะว่ามันช่วยทำให้ชีวิตเราง่าย%
ขึ้นเยอะเลยในการศึกษาหรือคำนวณคุณสมบัติเชิงอิเล็กทรอนิกส์ต่าง ๆ ของอะตอมอื่น ๆ ที่มีอิเล็กตรอนมากกว่า 1 ตัว

นอกจากพลังงานรวมแล้วเรายังสามารถคำนวณแรงที่กระทำต่ออะตอม (atomic force) แต่ละตัวในโมเลกุลได้ด้วย 
โดย atomic force นั้นก็คือเกี่ยวข้องกับการหาอนุพันธ์อันดับที่ 1 ของพลังงานรวมเทียบกับ atomic coordinate 
ของอะตอมนั้น ๆ ที่เราสนใจ ซึ่งเราสามารถหาได้โดยได้ผลเฉลยแบบแม่นตรง (exact solution) หรือจะใช้วิธี 
numerical method ก็ได้

สำหรับอัลกอริทึมที่เรานำมาใช้ในการปรับโครงสร้างของโมเลกุลหรือ minimization algorithms นั้นมีเยอะมาก ๆ 
โดยการปรับโครงสร้างของโมเลกุลนั้นก็คือเราทำการหาจุดที่ต่ำที่สุดในพื้นผิวพลังงานศักย์ซึ่ง 
ณ จุด ๆ นั้นโมเลกุลจะมีพลังงานที่ต่ำที่สุด อย่างไรก็ตามบ่อยครั้งเรามักจะมีโมเลกุลที่มี local minima หลายอัน 
สำหรับโมเลกุลเล็ก ๆ หรือโมเลกุลที่มีโครงสร้างไม่ซับซ้อนนั้นเราสามารถหาจุด minima 
ได้ง่ายเมื่อเทียบกับโมเลกุลที่มีขนาดใหญ่ พูดง่าย ๆ คือเมื่อโมเลกุลมีขนาดที่ใหญ่ขึ้นจำนวน local minima 
นั้นก็จะมากขึ้นตามไปด้วย โดยหนึ่งในตัวอย่างที่น่าสนใจก็คือสำหรับโมเลกุลหรือคลัสเตอร์ที่มี 55 อะตอมนั้นสามารถมีจำนวน 
minima ได้มากถึง $10^{21}$ เลยทีเดียว

\section{ความถี่ (Frequencies)}

Vibrational frequencies สามารถคำนวณได้จากพลังงานรวมเชิงอะตอม โดย vibration 
นั้นจะได้จากการคำนวณอนุพันธ์อันดับ 2 ของพลังงานรวมเทียบกับ atomic coordinates 
ของอะตอมคู่ในโมเลกุลซึ่งสามารถทำได้โดยใช้วิธี finite difference 

\section{Transition states}

ในความเป็นจริงนั้นคุณสมบัติต่าง ๆ ที่เกี่ยวข้องกับโครงสร้างของโมเลกุลนั้นสามารถคำนวณได้จาก atomic total energy 
โดยคุณสมบัติอันหนึ่งที่น่าสนใจมาก ๆ ก็คือ chemical reactivity หรือความว่องไวในการทำปฏิิกิริยา 
โดยค่าคงที่สมดุลของปฏิกิริยาและอัตราการเกิดปฏิริยานั้นสามารถคำนวณได้ง่าย ๆ แต่ว่าผลการคำนวณนั้นจะไม่ค่อยแม่นยำมากนัก 
สำหรับปฏิริยาเคมีนั้นสารตั้งต้นจะมีโครงสร้างของโมเลกุลที่ค่อนข้างแน่นอนเพราะว่ามักจะเป็นโครงสร้างที่มีพลังงานต่ำที่สุด 
แต่ว่าโครงสร้างของ transition states นั้นจะมี geometry ที่ต่างออกไป โดยกรณีทั่ว ๆ ไปนั้นโมเลกุลที่ไม่เป็นเส้นตรงนั้นจะมี 
degree of freedom คือ $3N - 6$ แต่ว่าจะมี transition state ได้เพียงแค่ geometry เดียวเท่านั้น 
(ระหว่าง reaction และ product) ถ้าหากว่า minima ของ reaction กับ product นั้นอยู่ห่างกันมาก ๆ 
ก็อาจจะมี transition state หลายอันก็ได้ซึ่งก็หมายความว่ปฏิกิริยานั้นมี Intermediates เกิดขึ้นระหว่าง 
reactant และ product นั่นเอง 

ประเด็นที่สำคัญอีกอันก็คือพลังงานที่เกี่ยวข้องกับปฏิกิริยาเคมีนั้นไม่ใช่พลังงานแต่ว่าเป็นพลังงานอิสระหรือ $F = H - TS$ 
ดังนั้นเราจึงต้องทำการประมาณค่า entropy ของระบบนั่นเอง สำหรับโมเลกุลที่อยู่ใน gas phase 
และโมเลกุลที่อยู่บน surface สามารถทำได้ไม่ยาก (เราแค่ต้องคำนวณ molecular vibration ออกมา) 
แต่ว่าโมเลกุลที่อยู่ในของเหลวหรือ liquid นั้นทำได้ยาก

อัลกอริทึมที่ใช้ในการคำนวณหาโครงสร้างของ transition state นั้นมีความซับซ้อนกว่าอัลกอริทึมของการทำ 
Minimization เยอะมาก ๆ โดยเรามีวิธีหลาย ๆ วิธีด้วยกันหนึ่งในนั้นก็ึคิอ Nudge Elastic Band (NEB) 
สำหรับวิธี NEB นั้นก็คือเราใช้ไอเดียของ reaction path ที่เป็นการเชื่อมโยงจุดหลาย ๆ จุดเข้าด้วยกัน 
ในทางปฏิบัตินั้นพลังงานที่เราคำนวณนั้นจะเป็นพลังงาน ไม่ใช่พลังงานอิสระ

ในการคำนวณ NEB นั้นเราจะคิดว่ามีสปริงที่เชื่อมระหว่างจุดโดยสปริงแต่ละอันจะมีค่าคงที่หรือความแข็งของสปริงเท่ากับค่า ๆ หนึ่ง 
คราวนี้เราจะต้องทำการ minimize พลังงานของ path ทั้งหมดโดยที่จุดที่ทั้งสองด้านนั้นจะถูกตรึงไว้ ตามหลักการนั้น 
(ตามอุดมคติ) path เริ่มต้นหรือ original path นั้นจะถูกปรับและขยับเข้าใกล้กับ path ที่ถูกต้องของปฏิกิริยามากขึ้นเรื่อย ๆ 
ถ้าหากว่าผู้อ่านสนใจศึกษารายละเอียดเเพิ่มเติมของ NEB ลองอ่านได้ที่ 
\url{http://www.openmx-square.org/tech_notes/NEB.pdf}

จริง ๆ แล้วสิ่งที่วิธี NEB ทำนั้นก็คือไม่ได้เป็นการหา transition state ที่ถูกต้องมากซะทีเดียวนักแต่ว่าเป็นการใช้ 
polynomial interpolation สำหรับการประมาณค่าเพื่อให้ผลที่สมเหตุสมผล 

โดยทั่วไปนั้นการหา transition state นั้นไม่ได้สามารถได้ง่าย ๆ จึงได้มีการพัฒนาวิธีการอื่น ๆ ขึ้นมารวมไปถึงปรับปรุงให้ 
NEB นั้นให้ผลที่ถูกต้องมากขึ้น แต่สิ่งที่เราควร note ไว้ก็คือ NEB นั้นให้พลังงานรวมออกมา ไม่ใช่พลังงานอิสระ 
ดังนั้นถ้าเราต้องการคำนวณพลังงานอิสระปฏิกิริยาเคมีเราจะต้องคำนวณ entropy ของระบบแยกต่างหาก 
แล้วก็วิธี NEB มีความสิ้นเปลืองและกินเวลามาก ๆ เพราะว่าจุดแต่ละจุดใน path ที่ถูกกำหนดขึ้นมานั้นจะเป็นการคำนวณด้วยวิธี 
quantum mechanics พร้อม ๆ กัน 

โดยสรุปก็คือ NEB นั้นเป็นหนึ่งในหลายวิธี ๆ ที่ใช้ในการหา transition state หรือ TS search แล้วก็การใช้ NEB 
ไม่ง่ายเท่าไหร่สำหรับผู้ที่เริ่มต้นศึกษาเคมีคำนวณแล้วก็พลังงานที่เราได้ออกมานั้นเป็นพลังงานรวม ไม่ใช่พลังงานอิสระ 

\section{Potential energy surface}

สิ่งที่ตรงไปตรงมาที่เราใช้ในการอธิบานพลังงานรวมของโมเลกุลมากที่สุดนั้นก็คือพื้นผิวพลังงานศักย์หรือ potential energy 
surface (PES) โดยกรณีทั่วไปเกือบทั้งหมดนั้นจริง ๆ แล้ว PES ก็คือ atomic total energy นั่นเอง 
แล้วก็เนื่องจากว่า total energy นั้นเป็นพารามิเตอร์หรือปริมาณที่มีมิติสูง (high dimensional) ดังนั้น PES approach 
นั้นจึงทำได้ค่อนข้างลำบากและยากในทางปฏิบัติแม้แต่กรณีง่าย ๆ เช่นโมเลกุลที่มีอะตอมเพียงแค่สองอะตอมเองก็ตาม 
โดย coordinate space นั้นมีจำนวนมิติถึง 6 มิติ ถ้าหากว่าเราใช้ 10 จุดในแต่ละมิติก็เท่ากับว่าเราจะต้องคำนวณทั้งหมด 
$10^{6}$ เพื่อทำการ map PES อันนี้ จริง ๆ แล้วตัวเลขมันฟังดูเยอะแต่ว่าในทางปฏิบัตินั้นทำได้แต่ว่าทำได้สำหรับระบบเล็กเท่านั้น 
(ไม่ได้สำหรับระบบใหญ่ ๆ) โดย PES แบบสมบูรณ์หรือ full PES นั้นไม่ค่อยได้รับความนิยมในการคำนวณมากนักเพราะว่า%
มันสิ้นเปลืองแต่ว่าก็พอมีประโยชน์อยู่บ้าง

\section{สรุป}

โดยสรุปก็คือ atomic total energy นั้นเป็นปริมาณที่มีประโยชน์มาก ๆ เพราะว่ามันเป็นกุญแจที่เราสามารถใช้ในการ%
คำนวณหาปริมาณอื่น ๆ ที่เราสนใจได้เยอะเลย เช่น โครงสร้างเชิงโมเลกุล, โหมดการสั่น, พลังงานรวมของระบบ 
รวมไปถึงการประมาณค่าของ reaction barriers แต่สิ่งที่ยังเป็นปัญหาหรือข้อจำกัดก็คือเรามักจะต้องการ free energy 
มากกว่า total energy นั่นเอง 
