% LaTeX source for ``Summary of Quantum Chemistry''
% Copyright (c) 2023 รังสิมันต์ เกษแก้ว (Rangsiman Ketkaew).

\chapter{Post Hartree-Fock}

\section{เกริ่นนำ}

ในปัจจุบันมีวิธีมีต่าง ๆ มากมายที่ทำให้วิธี HF นั้นสามารถคำนวณ Electronic structure ของโมเลกุลได้ถูกต้องมากขึ้น
เริ่มต้นผมขอพูดถึงคำว่า \enquote{Correlation Energy} หรือพลังงานสหสัมพันธ์ก่อน ซึ่งนิยามของ Correlation
Energy นั้นก็คือส่วนต่างระหว่างพลังงานรวมของโมเลกุล (Total Energy) กับพลังงาน Hartree-Fock (HF Energy)
หรือก็คือพลังงานที่หายไปที่ทำให้วิธี HF นั้นให้ผลการคำนวณที่ไม่ถูกต้องนั่นเอง โดยเขียนเป็นสมการได้ดังนี้

\begin{equation}
    E_{corr} = E_{tot} - E_{HF} < 0
\end{equation}

ตามที่เห็นก็คือว่านิยามดังกล่าวไม่ได้ช่วยให้เข้าใจอะไรเพิ่มขึ้นเลยแต่ว่าถ้าหากว่าเราสังเกตดี ๆ จะพบว่าถ้าเรามีวิธี%
ที่ดี ๆ บวกกับ Basis Set ที่สามารถคำนวณ Correlation Energy ได้แล้วล่ะก็เราจะสามารถทราบได้ว่า%
พลังงานที่ต่ำที่สุดนั้นก็คือค่าที่ดีที่สุดนั่นเอง ซึ่งก็คือไอเดียของ Variational Principle นั่นเอง

\section{Configuration Interaction}

วิธีการต่าง ๆ ที่ถูกพัฒนาขึ้นมาต่อยอดเพิ่มจากวิธี HF นั่นเรียกว่า Post Hartree-Fock (Post HF)
โดยวิธี Post HF อันหนึ่งที่มีแนวคิดเรียบง่ายที่สุดนั้นเรียกว่า Configuration Interaction (CI)
โดยวิธี CI นี้จะเป็นการสร้าง Wavefunction จาก Determinants หลาย ๆ อันดังนี้

\begin{equation}
    \Psi(r_{1}, \dots, r_{n})
    =
    a_{0} \Psi^{0}
    + \sum_{i,a} a^{a}_{i} \Psi^{a}_{i}
    + \sum_{ij,ab} a^{ab}_{ij} \Psi^{ab}_{ij}
    + \dots
\end{equation}

โดย Determinants อันใหม่ที่ถูกสร้างขึ้นมานี้นั้นได้มาจาก HF Orbitals แต่ว่าอิเล็กตรอนนั้นจะสามารถเข้าไปอยู่ใน
Excited States ได้ด้วย (มี Configuration เพิ่มขึ้นมาจากเดิมที่ Electron นั้นจะอยู่ใน Ground State
เท่านั้น) แล้วเราก็คำนวณอันตรกิริยาระหว่าง Configurations ทั้งหมดซึ่งก็คือ Interaction ดังนี้วิธีนี้จึงเรียกว่า
Configuration Interaction นั่นเอง

สำหรับการ interpret notation ของ $\Psi^{a}_{i}$ นั้นก็คือ Wavefunction อันนี้อธิบายอิเล็กตรอน 1
ตัวที่ถูกกระตุ้นจาก State $i$ ไปยัง State $a$

ส่วน $\Psi^{ab}_{ij}$ นั้นก็เหมือนกันคือเป็น Wavefunction ที่อธิบายอิเล็กตรอน 2 ตัวถูกกระตุ้นจาก
State $i$ กับ $j$ ไปยัง State $a$ กับ $b$ โดยเราจะเรียก Wavefunction อันนี้ว่า CISD (CI ที่มี Singlet
and Doublet Excitations) ส่วน $\Psi^{0}$ นั้นก็คือ Slater Determinant (หรือ HF Wavefunction)
นั่นเอง

ในการคำนวณของวิธี CI นั้น เราจะคอย keep ให้ HF Orbitals นั้นคงที่ (ไม่มีการเปลี่ยรแปลง) ส่วนตัวที่เรา%
จะทำการ Optimize นั่นก็คือ Coefficient $a$ โดยเมื่อเรานำ Variational Principle เข้ามาใช้นั้นจะได้ว่า

\begin{equation}
    E_{CIS} = \min_{a} ( \mel{\Psi^{0}}{H}{\Psi^{0}}
    + \sum_{i,a} a^{a}_{i} \mel{\Psi^{0}}{H}{\Psi^{a}_{i}} )
\end{equation}

\noindent หรืออาจจะใช้ Matrix formalism ก็ได้เช่นกัน ดังนี้

\begin{equation}
    H_{ij} = \mel{\Psi_{i}}{H}{\Psi_{j}}
\end{equation}

โดย $\Psi_{J}$ จะเป็น Determinant ของ Excited ที่ Level ไหนก็ได้ ในส่วนของการคำนวณนั้นเมทริกซ์%
อันนี้มีขนาดที่ใหญ่มากแต่ว่าสมาชิกส่วนใหญ่นั้นเป็น 0 ทำให้เมทริกซ์อันนี้เป็นแบบ Sparse Matrix
ดังนั้นเราจึงมีเทคนิคพิเศษบางอย่างที่สามารถเพิ่มความในการคำนวณที่มี Sparse Matrix เข้ามาเกี่ยวข้องได้

วิธี CISD นั้นนั้นสามารถให้ผลการคำนวณ Correlation Energy ที่ถูกต้องและสมเหตุสมผลสำหรับโมเลกุลที่มีขนาดเล็ก ๆ
เท่านั้น สำหรับโมเลกุลที่มีขนาดใหญ่นั้นวิธี CISD จะไม่ค่อยเวิร์คเท่าไหร่ (ไม่แง่ที่ว่าให้ผลการคำนวณที่ไม่ถูกต้องและ%
คำนวณได้ช้าหรือสิ้นเปลืองนั่นเอง) ดังนั้นเราจึงมีการเพิ่มเทอม On Top เข้าไปใน CISD เช่น เพิ่มเทอม Triplet
Excitation เข้าไปก็จะได้เป็นวิธี CISDT แต่ปัญหาคือวิธี CISDT นั้นสิ้นเปลืองมาก ๆ เพราะว่ามันมีจำนวนของ
Excitations เยอะมาก ๆ ที่จะต้องถูกคำนวณ ดังนั้นโดยทั่วไปแล้ววิธี CI จะสิ้นเปลืองมากขึ้นเรื่อย ๆ
เมื่อขนาดของโมเลกุลใหญ่ขึ้น โดยวิธี CI ที่ดีที่สุดนั้นเรียกว่า Full CI หรือ FCI ซึ่งเป็นการรวมการคำนวณ
Excitation ทั้งหมดที่เป็นไปได้เข้าไว้ด้วยกันซึ่งตามหลักการแล้ว FCI นั้นเป็น Exact Method ของทุก ๆ
ระบบเลยแต่ว่าในทางปฏิบัตินั้นเราสามารถทำ FCI ได้เฉพาะกับโมเลกุลเล็ก ๆ เท่านั้น

สำหรับวิธี FCI นั้นจำนวนของ Determinants ของ Excitation นั้นมีดังนั้น

\begin{equation}
    N_{det} = \binom{n}{k}^{2}
\end{equation}

\noindent โดย $n$ คือจำนวนของออร์บิทัล, $k$ คือจำนวนของอิเล็กตรอน ซึ่งจำนวนของ Determinant
นั้นจะเพิ่มขึ้นแบบ Exponentially และสำหรับกรณีที่ $n = 2k$ เมื่อ $k$ มีค่าเยอะ ๆ นั้น เราจะได้ว่า

\begin{equation}
    N_{det} = \binom{n}{k}^{2} \approx \frac{16^k}{k\pi}
\end{equation}

ถ้าสมมติว่าระบบเรามีจำนวน 10 Electrons ($k = 10$) จำนวนของ $N_{det}$ จะเท่ากับ $3.4 \times 10^{10}$
ซึ่งมันเป็นไปไม่ได้เลยที่จะคำนวณได้จริง ๆ

วิธีที่ใช้ในการคำนวณ Correlation Energy นั้นก็จะมี Computational scaling ที่แตกต่างกันออกไปแต่ส่วนใหญ่%
ก็มักจะสิ้นเปลืองทั้งนั้น แต่ในปัจจุบันเรามีโปรแกรม เช่น ORCA, Turbomole, Molcas, Molpro หรืออื่น ๆ
ที่ทำให้เราสามารถใช้วิธีเหล่านี้กับโมเลกุลที่มีขนาดใหญ่มากขึ้นได้แล้วก็สามารถทำได้ในเชิงปฏิบัติ (Practically Possible)
โดยเฉพาะกับวิธี MP2 และ CCSD(T) ที่สามารถถูกทำให้มีความสิ้นเปลืองของการคำนวณเทียบเท่าหรือพอ ๆ กับวิธี HF

\begin{table}[!htp]
    \Large
    \centering
    \begin{tabular}{lrr}\toprule
        Scaling & Method                \\\midrule
        N^4     & HF                    \\
        N^5     & MP2                   \\
        N^6     & MP3, CISD, CCSD       \\
        N^7     & MP4, CCSD(T)          \\
        N^8     & (MP5), CISDT, CCSDT   \\
        N^{10}  & (MP7), CISDTQ, CCSDTQ \\
        \bottomrule
    \end{tabular}
\end{table}

วิธี CI นั้นมีปัญหาอย่างนึงที่เรียกว่า Size Consistency Problem เช่น เรามีโมเลกุล He Dimer (He 2 อะตอม)
ซึ่ง He แต่ละอันอะตอมนั้นจะมีอิเล็กตรอน 2 ตัว ดังนั้นวิธี CISD นั้นจึงเป็น Exact Method สำหรับแค่อะตอม He 1
อะตอมเท่านั้นนั่นก็คือ CISD = FCI แต่เมื่อเรานำ CISD มาใช้กับโมเลกุล \ce{He2} เราจะพบว่าค่าพลังงานที่%
คำนวณได้จะยังไม่ใช่คำตอบแบบ Exact นั่นก็เพราะว่าเทอม Triple กับ Quadruple Excitations นั้นไม่มีซึ่งก็คือ
CISDTQ = FCI นั่นเอง

โดยทั่วไปแล้วความถูกต้องของวิธี CI นั้นจะลดลงตามขนาดที่ใหญ่มากขึ้นของโมเลกุลและการคำนวณทุกแบบนั้นก็เกี่ยวข้องกับ
size consistency ด้วย ซึ่งนี่ก็เป็นหนึ่งในปัญหาหลักสำหรับวิธี Post-HF แบบต่าง ๆ ในการคำนวณ binding Energy

\section{Active space}

Active Space SCF เป็นอีกวิธีหนึ่งสำหรับคำนวณ Correlation โดย assumption แรกของวิธีก็คือว่าจริง ๆ
แล้วเราไม่จำเป็นต้องนำ Excitation ทั้งหมดที่มีอยู่นั้นเข้ามาใช้ในการคำนวณ โดยเราสามารถกำหนด (Limit)
จำนวนของอิเล็กตรอนและจำนวนของ State ที่เราต้องการที่จะพิจารณา Configuration และ Excitation
ที่เกี่ยวข้องได้ โดยกระบวนการแบบนี้มีชื่อเรียกว่า Complete Active Space (CAS) และสำหรับโมเลกุลที่มี%
จำนวนอิเล็กตรอนและ State ที่เยอะนั้นจะมี Excitations เพียงแค่ $N$ Excitations เท่านั้นที่จะถูก%
พิจารณาและคำนวณซึ่งเราจะเรียกกระบวนการนี้ว่า Restricted Active Space (RAS)

นอกจากนี้แล้วเรายังสามารถทำการ Optimize ออร์บิทัลสำหรับ Excitation แต่ละอันได้ด้วยซึ่งเราเรียก%
กระบวนการนี้ว่า Multi Reference Method (MRSCF)\footnote{สำหรับวิธี CI นั้นออร์บิทัลจะไม่ถูก 
Optimized นะ} นี่จึงทำให้การคำนวณ total Electronic Energy ด้วยวิธี Active Space SCF
นั้นมีความถูกต้องและแม่นยำสูงมาก

โดยปกติแล้ววิธี CAS นั้นใช้ไม่ค่อยง่ายเท่าไหร่โดยเฉพาะสำหรับมือใหม่ที่เพิ่งเริ่มศึกษาเคมีคำนวณ
สาเหตุนึงก็คือว่าการเลือกพารามิเตอร์ต่าง ๆ สำหรับ CAS นั้นไม่มีสูตรตายตัว ขึ้นอยู่กับโมเลกุลแต่ละประเภท
แล้วก็มีปัญหาอื่น ๆ เช่น Numerical Problems ที่เกี่ยวข้องกับ Orbital Optimization ตามมาอีกด้วย
ดังนั้นวิธี CASSCF นั้นจึงไม่เหมาะคนที่เพิ่มเริ่มศึกษา

นอกจากนี้ยังมีเทคนิคพิศดารอีก เช่น เราสามารถนำ CI มารวมกับวิธี MR ได้ด้วยซึ่งก็จะได้เป็น MRCI
เช่น MRCISD (นำ CISD มารวมกับ MR) แต่ว่าไม่ค่อยมีคนใช้เท่าไหร่เพราะว่าเป็นวิธีที่ใช้งานยากแล้วก็%
สิ้นเปลืองการคำนวณระดับนึงเลย

\section{Coupled Cluster}

วิธี Coupled Cluster (CC) เป็นวิธีขั้นสูงวิธีหนึ่งที่ถูกพัฒนาขึ้นมาเพื่อ CI Wavefunctions
โดยวิธี CC จะทำการนิยามหรือ Represent Wavefunction ด้วยการใช้ Operator ดังต่อไปนี้

\begin{equation}
    \Psi_{CC} = e^{T} \Psi_{HF}
\end{equation}

\noindent โดยที่ $T = 1 + T_{1} + T_{2} + \dots + T_{n}$ โดยที่ $T_{n}$ นั้นคือ
Operator ที่จะทำการสร้าง Excited States ทั้งหมด $n$ อัน

ตัวอย่างเช่น

\begin{equation}
    T_{2} = \sum_{ij,ab} t_{ij,ab} \tau^{i}_{a} \tau^{j}_{b}
\end{equation}

จะเป็นการสร้าง Determinant ที่มีอิเล็กตรอนที่ถูกกระตุ้น (Excited Electrons) จำนวน 2 ตัว
ส่วน exponent ของ Operator นั้นมีความซับซ้อนทางคณิตศาสตร์มาก ๆ แต่ประเด็นหลักก็คือ
CC Wavefunctions นั้นจะมีความสมบูรณ์ (Completeness) มากกว่า CI Wavefunctions

โดย CC Wavefunction นั้นมี Contribution ดังนี้

\begin{equation}
    \hat{C} \Psi_{HF} = e^{T} \Psi_{HF}
\end{equation}

แล้วก็

\begin{gather}
    C_{0} = 1 \\
    C_{1} = T_{1} \\
    C_{2} = T_{2} + \frac{1}{2}T^{2}_{1} \\
    C_{3} = T_{3} + T_{1}T_{2} + \frac{1}{6}T^{3}_{1} \\
    C_{4} = T_{4} + T_{1}T_{3} + \frac{1}{2}T^{2}_{2}
    + \frac{1}{2}T^{2}_{1} T_{2} + \frac{1}{24}T^{4}_{1}
\end{gather}

ตัวอย่างเช่น ถ้าเราใช้วิธี Coupled Cluster with Singles and Doubles (CCSD) เราจะมี
Operator แค่ 2 อันเท่านั้นคือ $T_{1}$ กับ $T_{2}$ แล้วก็เราจะมีเทอมที่สูงขึ้นไปอีกในระดับ
$C_{3}$ กับ $C_{4}$ แล้วก็วิธีการหาคำตอบของ Amplitude นั้นจะต้องใช้ทริคเข้ามาช่วย
(ไม่ขอพูดถึง) แล้วก็วิธี CC นั้นไม่มีปัญหาเรื่อง Size Consistent ซึ่งดีกว่าวิธี CI

ในแง่ของการใช้งานจริงนั้นวิธี CC คือกล่องดำของเคมีควอนตัม ถ้าให้ขยายความ + ยกตัวอย่างก็คือ
CCSD นั้นเป็นวิธีที่โอเคเพราะให้ผลที่แม่นยำระดับนึง ส่วนวิธี CCSDT และ CCSDTQ นั้นเป็นวิธีที่%
โคตรดีแต่ว่าสิ้นเปลืองสุด ๆ เหมือนกันโดยเฉพาะ CCSDTQ ดังนั้นการนำวิธี Perturbation
เข้ามาใช้ในการจัดการ (treat) Triplet Excitation นั้นช่วยทำให้ประสิทธิภาพของวิธีดีขึ้น
ซึ่งเราเรียกว่า CCSD(T) ซึ่งมีความน่าเชื่อถือมาก ๆ

คำถามคือทำไม CCSD(T) จึงได้รับความนิยมมากกว่า CCSDT คำตอบก็คือ CCSD(T) นั้นจะให้ค่า
Correlation ที่ Overestimate เพียงแค่นิดหน่อยเท่านั้นเมื่อเทียบกับ CCSDT โดย CCSD(T)
ได้รับการยอมรับว่าเป็น \textbf{Golden Standard} ของเคมีควอนตัมเลยทีเดียว
นั่นหมายความว่าวิธี CCSD(T) นั้นมีความถูกต้องเกือบเทียบเท่ากับ FCI เลยแต่ว่าประหยัดเวลา%
ในการคำนวณเยอะกว่ามาก นอกจากนี้เมื่อเทียบกับวิธี MRSCF แล้ววิธี CC นั้นใช้งานง่ายเยอะกว่ามาก
อย่างไรก็ตาม CCSD(T) นั้นก็ไม่ได้ Cheap ซะทีเดียวดังนั้นจึงได้มีวิธีใหม่ที่ถูกพัฒนา On Top
ขึ้นมาเรียกว่า DLPNO-CCSD(T) สำหรับโมเลกุลที่มีขนาดใหญ่ไปจนถึงโคตรใหญ่ แล้วก็มีความ%
สิ้นเปลืองเทียบเท่ากับ HF ดังนั้นจึงไม่มีเหตุผลที่เราจะต้องไปใช้ CCSD

\section{Perturbation Method}

อีกวิธีหนึ่งที่สามารถคำนวณ Correlation ได้ก็คือ Perturbation Method ซึ่งมีไอเดียที่ง่ายมาก ๆ
และสามารถใช้ในการแก้ปัญหาทางเคมีควอนตัมอื่น ๆ ได้อีกด้วย เริ่มต้นเรากำหนดให้ Hamiltonian
$H^{(0)}$ ที่เราต้องการจะแก้นั้นมีการถูกรบกวนหรือ Perturb จากแหล่งกระตุ้นภายนอก โดย Full
Hamiltonian นั้นจะเป็นผลรวมของ Hamiltonian บวกกับ Perturbation ค่าน้อย ๆ
ที่มีการถูกสเกลด้วย $\lambda$ Parameter ซึ่ง Hamiltonian อันใหม่ที่ได้นี้ (Perturbed Hamiltonian) 
จะรวมผลของการตอบสนองของโมเลกุลต่อการกระตุ้น (Response) เข้าไปด้วยแล้ว โดยมีสมการทั่วไปดังนี้

\begin{equation}
    H = H^{(0)} + \lambda V
\end{equation}

เราสามารถเขียนพลังงานและ Wavefunction ได้ในรูปของอนุกรมกำลัง (Power Series)
ของ $\lambda$ ดังนี้

\begin{equation}
    E = E^{0} + \lambda E^{(1)} + \lambda^{2} E^{(2)} + \dots
\end{equation}

\noindent และ

\begin{equation}
    \Psi_{0} = \Psi_{0}^{0} + \lambda \Psi_{0}^{(1)}
    + \lambda^{2} \Psi_{0}^{(2)} + \dots
\end{equation}

คราวนี้เราสามารถทำการเขียนสมการตามขั้นลำดับสำหรับ Power ของ $\lambda$ แต่ละอันได้

\begin{gather}
    H^{(0)} \Psi_0^{(0)} = E^{(0)} \Psi_0^{(0)} \\
    H^{(0)} \Psi_0^{(1)}+V \Psi_0^{(0)}
    = E^{(0)} \Psi_0^{(1)}+E^{(1)} \Psi_0^{(0)} \\
    H^{(0)} \Psi_0^{(2)}+V \Psi_0^{(1)}
    = E^{(0)} \Psi_0^{(2)}+E^{(1)} \Psi_0^{(1)}+E^{(2)} \Psi_0^{(0)}
\end{gather}

โดยตรงนี้เราจะหมายเหตุเอาไว้ว่า Notation $\Psi_0^{(n)}$ นี้หมายความว่าเป็น Wavefunction
ของ Ground State Wave ที่มี Perturbation Level คือ $\mathrm{n}$
นอกจากนี้แล้ว Wavefunction ที่อยู่กันคนละ Level นั้นยังเป็น Orthogonal ซึ่งกันและกันด้วย
$\left\langle\Psi_0^{(n)} \mid \Psi_0^{(0)}\right\rangle = \delta_{n 0}$

แล้วเราก็สามารถแก้หาพลังงาน $E^{(1)}, \Psi^{(1)}, E^{(2)}, \dots$ ได้ด้วย

\begin{gather}
    E^{(1)} = \left\langle\Psi_0^{(0)} V \Psi_0^{(0)}\right\rangle \\
    E^{(2)} = \left\langle\Psi_0^{(0)} V \Psi_0^{(1)}\right\rangle \\
    \Psi_0^{(1)} = \sum_{n>0} \frac{\left\langle\Psi_n^{(0)} V \Psi_0^{(0)}\right\rangle}
    {E_0^{(0)}-E_n^{(0)}} \Psi_n^{(0)} \\
    E^{(2)} = \sum_{n>0} \frac{\left\langle\Psi_n^{(0)} V \Psi_0^{(0)}\right\rangle^2}
    {E_0^{(0)}-E_n^{(0)}}
\end{gather}

Note ไว้ว่า $\Psi^{(0)}_{n}$ คือ Non-Ground State Wavefunction ที่ Iteration Level = 0

Perturbtation Theory นั้นสามารถนำไปใช้ในการคำนวณว่า Correlation Energy ได้ ซึ่งเราเรียกวิธีนี้ว่า
Mollor-Presset Theory (MPn) ดังนั้นจริง ๆ แล้ว $H^{(0)}$ คือ HF และ Perturbation นั้นก็คือ

\begin{equation}
    V = \sum_{i < j} \frac{1}{r_{i} - r_{j}} - \sum_{nm} (J_{nm} - \frac{1}{2}K_{nm})
\end{equation}

โดยวิธี MPn นี้สุดยอดมาก ๆ เพราะว่ามันแก้ปัญหา HF Error ได้ โดย Second-order MP หรือ MP2 นั้นมี
Correlation ดังไปต่อนี้

\begin{equation}
    E^{(2)} = \sum_{i<j,a<b} \frac{[(ij|ab) - (ia|jb)]^{2}}
    {\epsilon_{i} + \epsilon_{j} - \epsilon_{a} - \epsilon_{b}}
\end{equation}

\noindent $i,j$ คือ Occupied State และ $a,b$ คือ Unoccupied State (State ในที่นี้ก็คือออร์บิทัลนั่นแหละ)
ดังนั้น MP2 ก็คือ $E^{(0)} + E^{(1)} + E^{(2)}$ (ในความเป็นจริงแล้ว $E^{(0)} + E^{(1)}$ ก็คือพลังงานของ
HF นั่นเอง)

นอกจาก MP2 แล้วเราสามารถเพิ่ม Correlation Terms ด้วย Order ที่สูงกว่านี้ได้ เช่น MP3, MP4, หรือ MP5
แต่เหตุผลที่ MP2 ได้รับความนิยมสูงมากนั่นก็เพราะว่ามันคำนวณได้เร็วแล้วก็เพียงพอแล้วต่อการทำ Correlation
ให้กับ HF ส่วนวิธี MP3 นั่นไม่ค่อยเวิร์คเท่าไหร่เพราะว่ามันไม่ได้ทำให้ MP2 นั้นมีความถูกต้องเพิ่มมากขึ้น ส่วนวิธี MP4
นั้นให้ผลที่ดีค่อนข้างถูกต้องกว่า MP2 เลยทีเดียวแต่ว่าเป็นวิธีที่สิ้นเปลืองมาก ๆ ดังนั้นยิ่งเป็น MPn ที่มี Correlation Terms
สูง ๆ นั้นไม่ได้ถูกใช้เยอะนักเพราะว่ามันไม่ได้ Converge นั่นเอง

รายละเอียดของ Convergence ของ MPn Series นั้นสามารถอ่านเพิ่มเติมได้จากหนังสือ
Molecular Electronic-Structure Theory แต่งโดย Trygve Helgaker, Poul Jørgensen, Jeppe Olsen

\section{RI-Approximation}

เทคนิค Resolution of Identity หรือ RI ถูกนำมาใช้ในการเพิ่มความเร็วในการคำนวณ MP2
แล้วก็ยังคงมีความถูกต้องเทียบเท่ากับ MP2 สำหรับโมเลกุลที่มีขนาดใหญ่มาก ๆ นั้น part ที่ใช้เวลาในการคำนวณนานมากที่สุดคือ
HF

ไอเดียของ RI-MP2 ก็คือการ Fit ฟังก์ชัน Double Wavefunction $(ia|$ กับฟังก์ชันเบสิสอันใหม่ $|P)$ หรือที่เราเรียกว่า
Auxiliary Basis ซึ่งเทอม $(ia|$ นี้ใช้เวลาในการคำนวนนานมาก การทำแบบนี้คือเป็นการ Approximate เทอม
Two-Electron Integral $(ij|ab)$ นั่นเอง

โดย Approximation ที่เราได้นั้นคือ

\begin{equation}
    (ij|ab) \approx \sum_{PQ} (ij|P) V_{pQ} (Q|ab)
\end{equation}

จะเห็นว่า Two-Electron Integral นั้นจะถูกแยกออกเป็นฟังก์ชันเบสิสย่อย 2 อัน ถึงแม้ว่า Approximation
อันนี้จะดูมีความซับซ้อนแต่ว่าในทางปฏิบัติแล้วใช้เวลาในการคำนวณเร็วกว่าคำนวณ $(ij|ab)$ เยอะมาก
รายละเอียดของวิธี RI นั้นซับซ้อนกว่านี้มากแต่ประเด็นหลักก็คือวิธี RI นั้นต้องการ Additional Basis

โดยสรุปก็คือ Perturbation Method สามารถช่วยแก้ปัญหาทางเคมีควอนตัมต่าง ๆ ได้เยอะเลย หนึ่งใน Applications
ก็คือวิธี CI กับ CC ซึ่ง Excitations ที่เป็น Highorder นั้นสามารถแก้ได้ด้วย Perturbation Theory
แล้วก็วิธี CCSD(T) คือวิธีมาตรฐาน (Golden Standard) ของเคมีควอนตัมเพราะว่ามันมี Scaling เทียบเท่ากับวิธี MP4

Basis Set ที่เหมาะสำหรับวิธี CI และ MPn เพื่อให้ค่าพลังงานที่คำนวณได้นั้น Converge ได้อย่างเหมาะสมและถูกต้อง%
นั้นปกติแล้วจะมีขนาดใหญ่กว่า Basis Set ที่ใช้สำหรับวิธี HF ดังนั้นนี่จึงเป็นจุดอ่อนอย่างนึงของวิธี Post-HF
ซึ่งมีความสิ้นเปลืองในการคำนวณสูงมาก ๆ เพราะว่า Time Scale นั้นแปรผันตามจำนวนของ Basis Set ที่ใช้
ดังนั้นยิ่งโมเลกุลมีขนาดใหญ่จำนวน Basis Set ก็เพิ่มมากขึ้นเยอะตามไปด้วยและทำให้การคำนวณ Post-HF
นั้นใช้เวลานานมากขึ้นนั่นเอง
