% LaTeX source for ``Summary of Quantum Chemistry''
% Copyright (c) 2023 รังสิมันต์ เกษแก้ว (Rangsiman Ketkaew).

\chapter{Post Hartree-Fock}

\section{เกริ่นนำ}

ในปัจจุบันมีวิธีมีต่าง ๆ มากมายที่ทำให้วิธี HF นั้นสามารถคำนวณ electronic structure ของโมเลกุลได้ถูกต้องมากขึ้น
เริ่มต้นผมขอพูดถึงคำว่า \enquote{correlation energy} หรือพลังงานสหสัมพันธ์ก่อน ซึ่งนิยามของ correlation
energy นั้นก็คือส่วนต่างระหว่างพลังงานรวมของโมเลกุล (total energy) กับพลังงาน Hartree-Fock (HF energy)
หรือก็คือพลังงานที่หายไปที่ทำให้วิธี HF นั้นให้ผลการคำนวณที่ไม่ถูกต้องนั่นเอง โดยเขียนเป็นสมการได้ดังนี้

\begin{equation}
    E_{corr} = E_{tot} - E_{HF} < 0
\end{equation}

ตามที่เห็นก็คือว่านิยามดังกล่าวไม่ได้ช่วยให้เข้าใจอะไรเพิ่มขึ้นเลยแต่ว่าถ้าหากว่าเราสังเกตดี ๆ จะพบว่าถ้าเรามีวิธี%
ที่ดี ๆ บวกกับ basis set ที่สามารถคำนวณ correlation energy ได้แล้วล่ะก็เราจะสามารถทราบได้ว่า%
พลังงานที่ต่ำที่สุดนั้นก็คือค่าที่ดีที่สุดนั่นเอง ซึ่งก็คือไอเดียของ variational principle นั่นเอง

\section{Configuration interaction}

วิธีการต่าง ๆ ที่ถูกพัฒนาขึ้นมาต่อยอดเพิ่มจากวิธี HF นั่นเรียกว่า Post Hartree-Fock (Post HF)
โดยวิธี Post HF อันหนึ่งที่มีแนวคิดเรียบง่ายที่สุดนั้นเรียกว่า Configuration Interaction (CI)
โดยวิธี CI นี้จะเป็นการสร้าง wavefunction จาก determinants หลาย ๆ อันดังนี้

\begin{equation}
    \Psi(r_{1}, \dots, r_{n})
    =
    a_{0} \Psi^{0}
    + \sum_{i,a} a^{a}_{i} \Psi^{a}_{i}
    + \sum_{ij,ab} a^{ab}_{ij} \Psi^{ab}_{ij}
    + \dots
\end{equation}

โดย determinants อันใหม่ที่ถูกสร้างขึ้นมานี้นั้นได้มาจาก HF orbitals แต่ว่าอิเล็กตรอนนั้นจะสามารถเข้าไปอยู่ใน 
excited states ได้ด้วย (มี configuration เพิ่มขึ้นมาจากเดิมที่ electron นั้นจะอยู่ใน ground state 
เท่านั้น) แล้วเราก็คำนวณอันตรกิริยาระหว่าง configurations ทั้งหมดซึ่งก็คือ interaction ดังนี้วิธีนี้จึงเรียกว่า 
configuration interaction นั่นเอง 

สำหรับการ interpret notation ของ $\Psi^{a}_{i}$ นั้นก็คือ wavefunction อันนี้อธิบายอิเล็กตรอน 1 
ตัวที่ถูกกระตุ้นจาก state i ไปยัง state a 

ส่วน $\Psi^{ab}_{ij}$ นั้นก็เหมือนกันคือเป็น wavefunction ที่อธิบายอิเล็กตรอน 2 ตัวถูกกระตุ้นจาก 
state i กับ j ไปยัง state a กับ b โดยเราจะเรียก wavefunction อันนี้ว่า CISD (CI ที่มี singlet 
and doublet excitations) ส่วน $\Psi^{0}$ นั้นก็คือ Slater determinant (หรือ HF wavefunction) 
นั่นเอง 

ในการคำนวณของวิธี CI นั้น เราจะคอย keep ให้ HF orbitals นั้นคงที่ (ไม่มีการเปลี่ยรแปลง) ส่วนตัวที่เรา%
จะทำการ optimize นั่นก็คือ coefficient $a$ โดยเมื่อเรานำ variational principle เข้ามาใช้นั้นจะได้ว่า 

\begin{equation}
    E_{CIS} = \min_{a} ( \mel{\Psi^{0}}{H}{\Psi^{a}_{i}} )
\end{equation}

\noindent หรืออาจจะใช้ matrix formalism ก็ได้เช่นกัน ดังนี้

\begin{equation}
    H_{IJ} = \mel{\Psi_{i}}{H}{\Psi_{J}}
\end{equation}

โดย $\Psi_{J}$ จะเป็น determinant ของ excited ที่ level ไหนก็ได้ ในส่วนของการคำนวณนั้นเมทริกซ์%
อันนี้มีขนาดที่ใหญ่มากแต่ว่าสมาชิกส่วนใหญ่นั้นเป็น 0 ทำให้เมทริกซ์อันนี้เป็นแบบ sparse matrix 
ดังนั้นเราจึงมีเทคนิคพิเศษบางอย่างที่สามารถเพิ่มความในการคำนวณที่มี sparse matrix เข้ามาเกี่ยวข้องได้

วิธี CISD นั้นนั้นสามารถให้ผลการคำนวณ correlation energy ที่ถูกต้องและสมเหตุสมผลสำหรับโมเลกุลที่มีขนาดเล็ก ๆ 
เท่านั้น สำหรับโมเลกุลที่มีขนาดใหญ่นั้นวิธี CISD จะไม่ค่อยเวิร์คเท่าไหร่ (ไม่แง่ที่ว่าให้ผลการคำนวณที่ไม่ถูกต้องและ%
คำนวณได้ช้าหรือสิ้นเปลืองนั่นเอง) ดังนั้นเราจึงมีการเพิ่มเทอม on top เข้าไปใน CISD เช่น เพิ่มเทอม triplet 
excitation เข้าไปก็จะได้เป็นวิธี CISDT แต่ปัญหาคือวิธี CISDT นั้นสิ้นเปลืองมาก ๆ เพราะว่ามันมีจำนวนของ 
excitations เยอะมาก ๆ ที่จะต้องถูกคำนวณ ดังนั้นโดยทั่วไปแล้ววิธี CI จะสิ้นเปลืองมากขึ้นเรื่อย ๆ 
เมื่อขนาดของโมเลกุลใหญ่ขึ้น โดยวิธี CI ที่ดีที่สุดนั้นเรียกว่า Full CI หรือ FCI ซึ่งเป็นการรวมการคำนวณ 
excitation ทั้งหมดที่เป็นไปได้เข้าไว้ด้วยกันซึ่งตามหลักการแล้ว FCI นั้นเป็น exact method ของทุก ๆ 
ระบบเลยแต่ว่าในทางปฏิบัตินั้นเราสามารถทำ FCI ได้เฉพาะกับโมเลกุลเล็ก ๆ เท่านั้น 

สำหรับวิธี FCI นั้นจำนวนของ determinants ของ excitation นั้นมีดังนั้น 

\begin{equation}
    N_{det} = \binom{n}{k}^{2}
\end{equation}

\noindent โดย $n$ คือจำนวนของออร์บิทัล, $k$ คือจำนวนของอิเล็กตรอน ซึ่งจำนวนของ determinant 
นั้นจะเพิ่มขึ้นแบบ exponentially และสำหรับกรณีที่ $n = 2k$ เมื่อ $k$ มีค่าเยอะ ๆ นั้น เราจะได้ว่า 

\begin{equation}
    N_{det} = \binom{n}{k}^{2} \approx \frac{16^k}{k\pi}
\end{equation}

ถ้าสมมติว่าระบบเรามีจำนวน 10 electrons ($k = 10$) จำนวนของ $N_{det}$ จะเท่ากับ $3.4 \times 10^{10}$ 
ซึ่งมันเป็นไปไม่ได้เลยที่จะคำนวณได้จริง ๆ

วิธีที่ใช้ในการคำนวณ correlation energy นั้นก็จะมี computational scaling ที่แตกต่างกันออกไปแต่ส่วนใหญ่%
ก็มักจะสิ้นเปลืองทั้งนั้น แต่ในปัจจุบันเรามีโปรแกรม เช่น ORCA, Turbomole, Molcas, Molpro หรืออื่น ๆ 
ที่ทำให้เราสามารถใช้วิธีเหล่านี้กับโมเลกุลที่มีขนาดใหญ่มากขึ้นได้แล้วก็สามารถทำได้ในเชิงปฏิบัติ (practically possible)
โดยเฉพาะกับวิธี MP2 และ CCSD(T) ที่สามารถถูกทำให้มีความสิ้นเปลืองของการคำนวณเทียบเท่าหรือพอ ๆ กับวิธี HF 

