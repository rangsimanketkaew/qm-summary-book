% LaTeX source for ``Summary of Quantum Chemistry''
% Copyright (c) 2023 รังสิมันต์ เกษแก้ว (Rangsiman Ketkaew).

\chapter{Density Functional Theory}

\section{เกริ่นนำ}

ทฤษฎีทางเคมีควอนตัมแบบดังเดิม (traditional quantum chemical methods) นั้นจะทำการจัดกับ
wavefunction โดยการ represent ด้วยฟังก์ชันแบบต่าง ๆ แล้วก็หาคำตอบของ correlation energy
ออกมาซึ่งมีความสิ้นเปลืองในเชิงการคำนวณสูงมากเพราะว่ามีขั้นตอนต่าง ๆ ที่ซับซ้อนและแทบจะนำไปใช้จริง%
ไม่ได้เลยกับระบบที่มีขนาดใหญ่

Density Functional Theory (DFT) หรือทฤษฎีฟังก์ชันแนลความหนาแน่นนั้นแตกต่างออกไป
ตัว DFT เองนั้นใช้แนวคิดที่ไม่ได้ผูกกับ wavefunction ตามที่เราได้ศึกษากันมาในที่แล้วอีกต่อไป
โดยมีหัวใจหลักก็คือจะมองว่า \enquote{พลังงานของระบบที่สภาวะพื้นนั้นจะขึ้นอยู่กับความหนาแน่น%
    ของอิเล็กตรอนเพียงเดียวเท่านั้น}

\begin{equation}
    \Psi(r_{1}, r_{2}, \dots, r_{N}) \rightarrow \Psi[\rho(r)]
\end{equation}

ซึ่ง wavefunction ในเวอร์ชันของ DFT นั้นง่ายกว่า wavefunction จริง ๆ เยอะเลย
แต่ปัญหาก็คือว่าเราไม่รู้สมการที่จะแก้ wavefunction อันนี้ออกมาได้ ไอเดียอันนี้เริ่มต้นมาจาก
Kohn, Hohenberg และ Sham (ตามปี ค.ศ. ที่ตีพิมพ์เปเปอร์ 1963, 1694) แล้วก็ทำให้วิธี
DFT นั้นได้รางวัลโนเบลสาขาเคมีในปี 1998\footnote{วิธี DFT ได้รางวัลโนเบลเพราะว่าไอเดียของวิธี
    ไม่ใช่ความแม่นยำของวิธีเพราะว่ายังมีวิธีอื่น ๆ อีกเยอะแยะมากมายที่แม่นยำกว่าวิธี DFT แต่เป็นเพราะว่า%
    มีคนนำไปใช้งานเยอะและใช้งานได้จริงกับระบบโมเลกุลตั้งแต่ขนาดเล็กไปจนถึงขนาดใหญ่เพราะว่าใช้เวลา%
    ในการคำนวณเร็วกว่าวิธีอื่น ๆ} แม้ว่าเราจะไม่รู้สมการที่ถูกต้องจริง ๆ ในการแก้ Kohn-Sham wavefunction
แต่ว่าเราก็มีการสร้าง approximation ขึ้นมา โดย Kohn กับ Sham ได้พิสูจน์และพบว่าเพียงแค่เราใช้
wavefunction ที่เป็น single Slater determinant มารวมกับสิ่งที่เรียกว่า Exchange-Correlation
นั้นก็ให้ค่าผลการคำนวณที่ถูกต้องแล้ว

พาร์ทที่น่าสนใจและเป็นปัญหามากที่สุดมาจนถึงปัจจุบันนี้ก็คือ Exchange-Correlation energy เพราะว่า ณ
ปัจจุบันนี้ยังไม่มีใครสามารถหาหน้าตาที่เป็น exact solution ของเทอม XC นี้ได้ ซึ่งฟังก์ชันนอล XC
ที่มีให้เราเลือกใช้กันตามโปรแกรมเคมีคำนวณต่าง ๆ ในปัจจุบันนี้เป็นเพียงแค่ approximation เท่านั้น

\section{Orbital-free DFT}

พลังงานของระบบสามารถเขียนเป็นสมการง่าย ๆ ได้ดังนี้

\begin{equation}
    E[\rho] = T_{s}[\rho] + W[\rho] - \int d^{3} r V(r,R_{I}) \rho(r)
\end{equation}

\noindent $T[\rho]$ คือพลังงานจลน์, $W[\rho]$ คือพลังงานของอันตรกิริยาระหว่างอิเล็กตรอน
(exact electron-electron interaction energy) ซึ่งมีหน้าตาดังนี้

\begin{gather}
    T[\rho] = -\sum_{i} \mel{\psi}{\nabla^{2}_{i}}{\psi} \\
    W[\rho] = \int \frac{\rho(r)\rho(r)'}{|r-r'|} d^{3} r d^{3} r' + E_{XC}(\rho)
\end{gather}

เทอมที่เราไม่รู้นั้นก็คือ $T[\rho]$ และ  $E_{XC}(\rho)$ แต่ว่าเรามี approximation สำหรับทั้งสองเทอมนี้
(อย่างไรก็ตาม approximation ของ $T[\rho]$ ที่เรามีนั้นก็ไม่ค่อยให้ผลการคำนวณที่ถูกต้องเท่าไหร่)

สมการด้านบนนี้เป็นของวิธีที่เรียกว่า density-only DFT ก็คือเป็น DFT ที่ขึ้นอยู่กับ density เท่านั้น
หรือมีชื่อเรียกอีกอย่างในวงการว่า orbital-free DFT ซึ่งให้ผลการคำนวณที่ไม่ถูกต้องเลย

\section{Kohn-Sham DFT}

ลำดับถัดมาคือเป็น DFT scheme อันนึงที่ถูกเสนอโดย Kohn กับ Sham ซึ่งเขาทั้งสองคนนั้นมี assumption
ว่าจริง ๆ แล้วจะมีระบบที่อิเล็กตรอนไม่ขึ้นต่อกัน (non-interaction system) ที่สามารถมี density
ที่ถูกต้องได้ (และระบบที่มี density ถูกต้องนั้นก็จะให้ค่าพลังงานของระบบที่ถูกต้องด้วยเช่นกัน)

สำหรับ non-interaction system นั้น เราจะเริ่มจาก Slater determinant ของ wavefunction
ก่อน ดังนี้

\begin{equation}
    \Psi_{KS} = det[\phi_{1,KS}, \phi_{2,KS}, \dots, \phi_{N,KS}]
\end{equation}

\noindent และมี density คือ

\begin{equation}
    \rho(r) = \sum^{N}_{n=1} \psi_{n,KS}^{\ast}(r) \psi_{n,KS}(r)
\end{equation}

\noindent ซึ่งเราสามารถสร้าง Kohn-Sham equation โดยใช้ Kohn-Sham orbitals ได้ ดังนี้

\begin{equation}
    \left[-\frac{\hbar^2}{2 m_e} \sum_i \nabla_i^2
        + V_{K S}[\rho ; R](r)\right]
    \varphi_n^{K S}\left(r_i ; R_I\right)
    = E_n \varphi_n^{K S}\left(r_i ; R_I\right)
\end{equation}

\noindent สมการด้านบนนี้ดูแล้วอุรุงตุงนังไปหมด เราค่อย ๆ มาทำความเข้าใจไปพร้อม ๆ กัน ถ้าเอาแบบง่าย ๆ
ก็คือก้อนด้านซ้ายสุดที่อยู่ใน $[ ]$ ก็คือ Hamiltonian ส่วน $\varphi$ คือ wavefunction และ $E$
คือพลังงานของระบบ (คำตอบที่เราต้องการหาจาก wavefunction)

สำหรับ Hamiltonian นั้นจะมีเทอมแรกทางซ้ายคือเทอมพลังงานจลน์ ส่วนเทอมที่สองคือพลังงานศักย์

ประเด็นก็คือว่าเราไม่รู้ว่า $\mathrm{V}_{\mathrm{KS}}(\mathrm{r})$ มีหน้าตาเป็นยังไง (unknown potential)
โดย $\mathrm{r}$ คือตำแหน่งของอิเล็กตรอนและ $\mathrm{R}$ คือตำแหน่งของนิวเคลียส
ซึ่งถ้าเขียนสมการด้านบนนี้แบบจัดเต็มเวอร์ ๆ ก็จะมีหน้าตาแบบนี้

\begin{equation}
    \left[-\sum_i \frac{\hbar^2}{2 m_i} \nabla_i^2+2 \int d^3 r_j
        \frac{\rho\left(r_j\right)}{\left|r_i-r_j\right|}
        -\sum_{i J} \frac{Z_I e^2}{4 \pi \varepsilon_0\left|r_i-R_J\right|}
        + V_{X C}[\rho](r)\right] \varphi_n^{K S}\left(r_i ; R_I\right)
    = E \varphi_n^{K S}\left(r_i ; R_I\right)
\end{equation}

สังเกตว่าสิ่งที่เพิ่มขึ้นมาก็คือเราทำการแบ่ง $V_{K S}[\rho ; R](r)$ ออกมาเป็น potential ย่อย ๆ
ตามพิกัดตำแหน่งของอิเล็กตรอนและนิวเคลียส ซึ่งจะเห็นเรารู้หน้าตาของ $V_{X C}(R)$ (เพราะเรารู้ตำแหน่งที่แน่นอนของอะตอมไง)
แต่ว่ายังมีเทอม $V_{X C}(r)$ ที่เรายังไม่รู้อีก แต่ว่าเราค่อนข้างโชคดีนิดนึงที่เทอมที่เราไม่รู้นี้มันไม่ได้ใหญ่มาก%
และเราสามารถใช้ approximation บางอย่างได้ในการหาเทอม potential ที่ขึ้นกับตำแหน่งของอิเล็กตรอนนี้ได้

\section{Approximation ของ $V_{X C}$}

เราต้องการหา XC energy และ potential ซึ่งก็คือ

\begin{equation}
    E_{XC} = \int V_{X C}(r) \rho(r)
\end{equation}

\noindent และ

\begin{equation}
    V_{X C}(r) = \frac{\delta E_{X C}}{\delta \rho(r)}
\end{equation}

\subsection{LDA}

Approximation อันแรกเลยที่ถูพัฒนาขึ้นคือ Local density approximation (LDA) ซึ่งสามารถคำนวณพลังงานรวม%
ของระบบ homogeneous electron gas ได้แม่นยำระดับหนึ่งเลยทีเดียว ซึ่งเรารู้ $E_{XC, homog}(\rho)$
นั่นเอง

ตัวพลังงานของ LDA นั้น ($E^{LDA}_{XC}$) มีสมการดังนี้

\begin{equation}
    E^{LDA}_{XC} = \int E_{XC,homog} (\rho(r)) * \rho(r)d^{3}r
\end{equation}

โดยเทอม exchange ของ LDA นั้นหาได้ง่ายแต่ว่าเทอม correlation นั้นซับซ้อนและถ้าเป็นการหา approximation
สำหรับ correlation นั้นถ้าทำแบบ traditionally แล้วจะอาศัยการ fit กับการคำนวณ Quantum Monte Carlo

\subsection{GGA}

เราสามารถทำให้ LDA นั้นมีประสิทธิภาพมากขึ้นเพื่อให้คำนวณได้ถูกต้องมากขึ้นโดยการรวม gradient ของความหนาแน่นเข้าไปด้วย
ดังนี้

\begin{equation}
    E^{GGA}_{XC} = E^{LDA}_{XC} + \int E_{XC} (\rho(r), \nabla\rho) * \rho(r)d^{3}r
\end{equation}

สมการด้านบนนี้ดูผ่าน ๆ เหมือนจะไม่ซับซ้อนแต่จริง ๆ ซับซ้อนมาก ๆ และการหา approximation ของ GGA
นั้นไม่ใช่เรื่องง่ายเลย หนึ่งใน approximation ที่ง่ายที่สุดของ GGA นั้นถูกเสนอโดย Axel Becke (ปี 1986)
ซึ่งมีสมการดังนี้

\begin{gather}
    E^{GGA}_{XC} = \int E_{XC} (\rho) * f_{x}(\xi) d^{3}r \\
    \xi = \frac{(\nabla \rho)^{2}}{(2 k_{F} \rho)^{2}}
    \quad , \quad k_{F} = (3\pi^{2}\rho)^{\frac{1}{3}} \\
    f^{B86}_{x}(\xi) = 1 + \frac{a\xi}{1 + b\xi}
\end{gather}

\noindent โดยที่ $a = 0.2351$ และ $b = 0.24308$ โดย GGA functional อันนี้มีโค้เนมที่เรียกกันในวงการคือ B86

หลังจากนั้นอีก 2 ปี (ปี 1988) Becke ก็ได้เสนอ GGA functional ที่ปรับปรุงมาจาก B86 อีกแล้ว functional
อันใหม่นี้ว่า B88 ซึ่งมีฟังก์ชันของ $f_{x}(\xi)$ ที่ซับซ้อนมากขึ้น ดังนี้ 

\begin{equation}
    f^{B88}_{x}(\xi) = 
    1 + \frac{a\xi}{1 + b\sqrt{\xi} arsh (2(6\pi^{2})^{\frac{1}{3}} \sqrt{\xi })}
\end{equation}

\noindent แล้วก็ค่าคงที่เปลี่ยนเป็น $a = 0.2743$ และ $b = \frac{9a}{4\pi}$

ส่วน correlation นั้นจริง ๆ แล้วมี functional model หลายอันมาก ๆ ส่วนอันที่ถูกใช้เยอะที่สุดก็คือ PBE 
(ย่อมาจากนามสกุลของนักเคมีทฤษฎี 3 คนคือ Perdew-Burke-Ernzerhof) ซึ่งหน้าตาของสมการนั้นขอไม่โชวน์ให้ดูนะ
(เพราะขี้เกียจพิมพ์ครับ) โดยผู้อ่านสามารถดูหน้าตาสมการได้ที่ Wikipedia หรือหนังสือ electronic structure ทั่วไป 

ส่วน approximation อื่น ๆ สำหรับ DFT ในตระกูลนี้นั้นยังมีอีกเยอะเลย มีหลาย ๆ ที่ได้จามาจากการ fit กับข้อมูลการทดลอง 
(empirical parameters) เช่น B86 หรือ HCTH 

Scaling ของ GGA ก็ดีด้วย ประมาณ $N^{3}$ เท่านั้นเองซึ่งคำนวณได้เร็วกว่า HF มาก ๆ

ดูรายละเอียดเพิ่มเติมที่ \url{https://tddft.org/programs/libxc/}

\subsection{Meta-GGA}

Meta-GGA functional คือเป็นการยกระดับ GGA ขึ้นมาโดยการใช้ข้อมูลของอนุพันธ์อันดับ 2 ของความหนาแน่น (second 
derivative of the density) และ/หรือ kinetic energy density เข้ามาช่วยด้วย 

Meta-GGA functional เช่น TPSS, M06-L, rev-TPSS, แล้วก็ SCAN ให้ผลการคำนวณที่ถูกต้องกว่า GGA 
เพราะว่ามันสามารถ treat ประเภทของพันธะเคมีในโมเลกุลที่แตกต่างกันได้ เช่น covalent bond, metallic bond, 
weak bond) 

\subsection{Hybrid functional}

ไอเดียของ hybrid functional ก็ตามชื่อเลยคือเป็นการนำ HF มาผสมกับ DFT ซึ่งให้ผลการคำนวณที่ถูกต้องมาก ๆ 
โดย functionals ในตำนานยอดฮิตที่ผู้คนต่างก็ยังใช้กันมานานก็คือ B3LYP กับ PBE0 

\begin{equation}
    E^{hybrid}_{XC} = a E^{HF}_{X} + (1 - a) E^{GGA}_{X} + b E^{LDA}_{c}
\end{equation}

\section{ปัญหาของ DFT}

จุดอ่อนหลักของ DFT ก็คือเราไม่มี systematic way ในการปรับปรุง XC potential นั่นคือ functionals 
แต่ละอันที่แต่ละเจ้าเสนอมานั้นมันได้มาจากทฤษฎีที่แตกต่างกัน เห็นได้จากจำนวนของ functionals 
ที่มีอยู่ในปัจจุบันนี้ที่มีจำนวนมากมายมหาศาล ซึ่งแทบจะเป็นไปไม่ได้เลยที่ XC functional ในอุดมคติหรือ 
universal XC functional นั้นจะถูกหาเจอ\footnote{Universal functional เป็น functional 
ที่มีสรรพคุณครอบจักรวาลก็คือสามารถนำไปใช้ในการคำนวณกับระบบไหนก็ได้ที่ต้องการ ซึ่งมันคือ holy grail 
ของ DFT เลยก็ว่าได้}

ความถูกต้องของ DFT model นั้นเรียงตามประเภทของ approximation functional ได้ดังนี้ 

LDA (ห่วย) < GGA < meta-GGA \~ hybrid functionals 

\noindent และ 

GGA < GGA + dispersion correction 

\section{Self-interaction error}

อีกหนึ่งปัญหาคลาสสิคของ DFT ก็คือ self-interaction error ก็คือว่าใน DFT นั้น อิเล็กตรอนแต่ละตัวนั้น%
มันจะสัมผัสหรือ feel ได้ถึงตัวเอง ซึ่งกลายเป็นมันมี error ที่เกิดจาก interaction ระหว่างอิเล็กตรอนตัวมันเอง 
ดังนั้นการแก้ปัญหาดังกล่าวก็คือการตัดหรือลบ orbital ที่ขึ้นอยู่กับ self-interaction ออกไป 

โดยการทำ correction แบบนี้นั้นมีความซับซ้อนมากเพราะว่ามันเกี่ยวข้องกับ orbital แทนที่จะเป็น density 
ซึ่งกลายเป็นว่าการที่เราแบบนี้มันทำให้ทฤษฎีอันนี้มันไม่ใช่ DFT อีกต่อไปแล้ว นอกจากนี้ยังมีปัญหาอื่น ๆ ตามมาอีก เช่น 
ปัญหาทางด้านการคำนวณเพราะว่าการใช้ self-interaction correction นั้นมันทำให้การ converge 
ของพลังงานนั้นยากขึ้น (ค่าพลังงานที่ได้ออกจากจึงไม่ถูกต้อง)

นั่นจึงทำให้วิธี HF นั้นมีข้อได้เปรียบกว่า DFT ในแง่ที่ว่ามันไม่มีปัญหาเรื่อง self-interaction จึงเป็นเหตุผลที่ว่าทำไม 
hybrid functional นั้นให้ผลการคำนวณที่ถูกต้องการ GGA models 
