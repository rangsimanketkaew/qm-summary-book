% LaTeX source for ``Summary of Quantum Chemistry''
% Copyright (c) 2023 รังสิมันต์ เกษแก้ว (Rangsiman Ketkaew).

\chapter{Density Functional Theory}

\section{เกริ่นนำ}

ทฤษฎีทางเคมีควอนตัมแบบดั้งเดิม (Traiditional Quantum Chemical Methods) นั้นจะแสดง (Represent) ด้วยฟังก์ชันแบบต่าง ๆ แล้วก็หาคำตอบของ Correlation Energy ออกมา วิธีเหล่านี้นั้นโดยปกติแล้วจะมีความความสิ้นเปลืองในเชิงการคำนวณสูงมาก (Computational Demanding) เพราะว่ามีขั้นตอนต่าง ๆ ที่ซับซ้อนและแทบจะนำไปใช้จริงกับระบบทางเคมีไม่ได้เลย เช่น โมเลกุลที่มีจำนวนอะตอมหลักสิบขึ้นไป

Density Functional Theory (DFT) หรือทฤษฎีฟังก์ชันแนลความหนาแน่นนั้นแตกต่างออกไปจากทฤษฎีเดิม ๆ นั่นก็คือ DFT เองนั้นใช้แนวคิดที่ไม่ได้ผูกกับ Wavefunction ตามที่เราได้ศึกษากันไปแล้วในหัวข้อก่อนหน้านี้ โดยหลักการอันแรกที่เรียกได้ว่าเป็นจุดเนิดของ DFT นั้นมีอยู่ว่า \enquote{พลังงานของระบบที่สภาวะพื้นนั้นจะขึ้นอยู่กับความหนาแน่นของอิเล็กตรอนเพียงเดียวเท่านั้น}
%
\begin{equation}
    \Psi(r_{1}, r_{2}, \dots, r_{N}) \rightarrow \Psi[\rho(r)]
\end{equation}
%
\noindent ซึ่ง Wavefunction ในเวอร์ชันของ DFT นั้นง่ายกว่า Wavefunction ของ Schr\"{o}dinger เยอะเลย แต่ปัญหาก็คือว่าเราไม่รู้สมการที่จะแก้ Wavefunction ของ DFT อันนี้ออกมาได้ ไอเดียอันนี้เริ่มต้นมาจาก Kohn, Hohenberg และ Sham (ตามปี ค.ศ. ที่ตีพิมพ์เปเปอร์ 1963, 1694) ซึ่งก็ทำให้วิธี DFT นั้นได้รางวัลโนเบลสาขาเคมีในปี 1998\footnote{วิธี DFT ได้รางวัลโนเบลเพราะว่าไอเดียของวิธีไม่ใช่ความแม่นยำของวิธีเพราะว่ายังมีวิธีอื่น ๆ อีกเยอะแยะมากมายที่แม่นยำกว่าวิธี DFT แต่เป็นเพราะว่ามีคนนำไปใช้งานเยอะและใช้งานได้จริงกับระบบโมเลกุลตั้งแต่ขนาดเล็กไปจนถึงขนาดใหญ่เพราะว่าใช้เวลาในการคำนวณเร็วกว่าวิธีอื่น ๆ} แม้ว่าเราจะไม่รู้สมการที่ถูกต้องจริง ๆ ในการแก้ Kohn-Sham Wavefunction แต่ว่าเราก็มีการสร้าง Approximation ขึ้นมา โดย Kohn กับ Sham ได้พิสูจน์และพบว่าเพียงแค่เราใช้ Wavefunction ที่เป็น Single Slater Determinant มารวมกับสิ่งที่เรียกว่า Exchange-Correlation นั้นก็ให้ค่าผลการคำนวณที่ถูกต้องแล้ว

พาร์ทที่น่าสนใจและเป็นปัญหามากที่สุดมาจนถึงปัจจุบันนั้นก็คือ Exchange-Correlation Energy (\textit{Exchange = แลกเปลี่ยน} และ \textit{Correlation = สหสัมพันธ์} เรียกสั้น ๆ ว่า XC)  เพราะว่า ณ ปัจจุบันนี้ยังไม่มีใครสามารถหาหน้าตาที่เป็น Exact Solution ของเทอม XC นี้ได้ ซึ่งฟังก์ชันนอล XC ที่มีให้เราเลือกใช้กันตามโปรแกรมเคมีคำนวณต่าง ๆ ในปัจจุบันนี้เป็นเพียงแค่ Approximation เท่านั้น

\section{Orbital-Free DFT}

พลังงานของระบบสามารถเขียนเป็นสมการง่าย ๆ ได้ดังนี้
%
\begin{equation}
    E[\rho] = T_{s}[\rho] + W[\rho] - \int d^{3} r V(r,R_{I}) \rho(r)
\end{equation}
%
\noindent $T[\rho]$ คือพลังงานจลน์, $W[\rho]$ คือพลังงานของอันตรกิริยาระหว่างอิเล็กตรอน (Exact Electron-Electron Interaction Energy) ซึ่งมีหน้าตาดังนี้
%
\begin{gather}
    T[\rho] = -\sum_{i} \mel{\psi}{\nabla^{2}_{i}}{\psi} \\
    W[\rho] = \int \frac{\rho(r)\rho(r)'}{|r-r'|} d^{3} r d^{3} r' + E_{XC}(\rho)
\end{gather}
%
เทอมที่เราไม่รู้นั้นก็คือ $T[\rho]$ และ  $E_{XC}(\rho)$ แต่ว่าเรามี Approximation สำหรับทั้งสองเทอมนี้ (อย่างไรก็ตาม Approximation ของ $T[\rho]$ ที่เรามีนั้นก็ไม่ค่อยให้ผลการคำนวณที่ถูกต้องเท่าไหร่)

สมการด้านบนนี้เป็นของวิธีที่เรียกว่า Density-Only DFT ก็คือเป็น DFT ที่ขึ้นอยู่กับ Density เท่านั้น หรือมีชื่อเรียกอีกอย่างในวงการว่า Orbital-Free DFT ซึ่งให้ผลการคำนวณที่ไม่ถูกต้องเลย

\section{Kohn-Sham DFT}

ถัดมาคือ DFT Scheme รูปแบบหนึ่งที่ถูกเสนอโดย Kohn กับ Sham ซึ่งเขาทั้งสองคนนั้นมี Assumption ว่าจริง ๆ แล้วจะมีระบบที่อิเล็กตรอนไม่ขึ้นต่อกัน (Non-Interaction System) ที่สามารถมี Density ที่ถูกต้องได้ (และระบบที่มี Density ถูกต้องนั้นก็จะให้ค่าพลังงานของระบบที่ถูกต้องด้วยเช่นกัน)

สำหรับ Non-Interaction System นั้น เราจะเริ่มจาก Slater Determinant ของ Wavefunction ก่อน ดังนี้
%
\begin{equation}
    \Psi_{KS} = det[\phi_{1,KS}, \phi_{2,KS}, \dots, \phi_{N,KS}]
\end{equation}
%
\noindent และมี Density คือ
%
\begin{equation}
    \rho(r) = \sum^{N}_{n=1} \psi_{n,KS}^{\ast}(r) \psi_{n,KS}(r)
\end{equation}
%
\noindent ซึ่งเราสามารถสร้าง Kohn-Sham Equation โดยใช้ Kohn-Sham Orbitals ได้ ดังนี้
%
\begin{align}
    \left[-\frac{\hbar^2}{2 m_e} \sum_i \nabla_i^2
        + V_{K S}[\rho ; R](r)\right] &
    \varphi_n^{K S}\left(r_i ; R_I\right) \nonumber \\
    &= E_n \varphi_n^{K S}\left(r_i ; R_I\right)
\end{align}
%
\noindent สมการด้านบนนี้ดูแล้วอิรุงตุงนัง (ยุ่งเหยิง) ไปหมด เราค่อย ๆ มาทำความเข้าใจไปพร้อม ๆ กัน ถ้าเอาแบบง่าย ๆ ก็คือก้อนด้านซ้ายสุดที่อยู่ใน $[\,\,]$ ก็คือ Hamiltonian ส่วน $\varphi$ คือ Wavefunction และ $E$ คือพลังงานของระบบ (คำตอบที่เราต้องการหาจาก Wavefunction)

สำหรับ Hamiltonian นั้นจะมีเทอมแรกทางซ้ายคือเทอมพลังงานจลน์ ส่วนเทอมที่สองคือพลังงานศักย์

ประเด็นก็คือว่าเราไม่รู้ว่า $\mathrm{V}_{\mathrm{KS}}(\mathrm{r})$ มีหน้าตาเป็นยังไง (Unknown Potential) โดย $\mathrm{r}$ คือตำแหน่งของอิเล็กตรอนและ $\mathrm{R}$ คือตำแหน่งของนิวเคลียส ซึ่งถ้าเขียนสมการด้านบนนี้แบบจัดเต็มเวอร์ ๆ ก็จะมีหน้าตาแบบนี้
%
\begin{align}
    \biggl[ & -\sum_i \frac{\hbar^2}{2 m_i} \nabla_i^2+2 \int d^3 r_j
        \frac{\rho\left(r_j\right)}{\left|r_i-r_j\right|} \nonumber \\
        & -\sum_{i J} \frac{Z_I e^2}{4 \pi \varepsilon_0\left|r_i-R_J\right|}  
        + V_{X C}[\rho](r) \biggr] \varphi_n^{K S}\left(r_i ; R_I\right) \nonumber \\
    & = E \varphi_n^{K S}\left(r_i ; R_I\right)
\end{align}

สังเกตว่าสิ่งที่เพิ่มขึ้นมาก็คือเราทำการแบ่ง $V_{K S}[\rho ; R](r)$ ออกมาเป็น Potential ย่อย ๆ ประกอบไปด้วย Potential ที่ขึ้นกับพิกัดตำแหน่งของอิเล็กตรอนและ Potential ที่ขึ้นกับพิกัดตำแหน่งของนิวเคลียส ซึ่งจะเห็นเรารู้หน้าตาของ $V_{X C}(R)$ (เพราะเรารู้ตำแหน่งที่แน่นอนของอะตอมไง) แต่ว่ายังมีเทอม $V_{X C}(r)$ ที่เรายังไม่รู้อีก แต่ว่าเราค่อนข้างโชคดีนิดนึงที่เทอมที่เราไม่รู้นี้มันไม่ได้ใหญ่มากและเราสามารถใช้ Approximation บางอย่างได้ในการหาเทอม Potential ที่ขึ้นกับตำแหน่งของอิเล็กตรอนนี้ได้

\section{Approximation ของ $V_{X C}$}

เราต้องการหา XC Energy และ Potential ซึ่งก็คือ
%
\begin{equation}
    E_{XC} = \int V_{X C}(r) \rho(r)
\end{equation}
%
\noindent และ
%
\begin{equation}
    V_{X C}(r) = \frac{\delta E_{X C}}{\delta \rho(r)}
\end{equation}

\subsection{LDA}

Approximation อันแรกเลยที่ถูกพัฒนาขึ้นคือ Local Density Approximation (LDA) ซึ่งสามารถคำนวณพลังงานรวมของระบบ Homogeneous Electron gas ได้แม่นยำระดับหนึ่งเลยทีเดียว ซึ่งเรารู้ $E_{XC, homog}(\rho)$ นั่นเอง

ตัวพลังงานของ LDA นั้น ($E^{LDA}_{XC}$) มีสมการดังนี้
%
\begin{equation}
    E^{LDA}_{XC} = \int E_{XC,homog} (\rho(r)) * \rho(r)d^{3}r
\end{equation}
%
โดยเทอม Exchange ของ LDA นั้นหาได้ง่ายแต่ว่าเทอม Correlation นั้นซับซ้อนและถ้าเป็นการหา Approximation สำหรับ Correlation นั้นถ้าทำแบบ Traiditionally แล้วจะอาศัยการ fit กับการคำนวณ Quantum Monte Carlo

\subsection{GGA}

เราสามารถทำให้ LDA นั้นมีประสิทธิภาพมากขึ้นเพื่อให้คำนวณได้ถูกต้องมากขึ้นโดยการรวม gradient ของความหนาแน่นเข้าไปด้วย ดังนี้
%
\begin{equation}
    E^{GGA}_{XC} = E^{LDA}_{XC} + \int E_{XC} (\rho(r), \nabla\rho) * \rho(r)d^{3}r
\end{equation}
%
สมการด้านบนนี้ดูผ่าน ๆ เหมือนจะไม่ซับซ้อนแต่จริง ๆ ซับซ้อนมาก ๆ และการหา Approximation ของ GGA นั้นไม่ใช่เรื่องง่ายเลย หนึ่งใน Approximation ที่ง่ายที่สุดของ GGA นั้นถูกเสนอโดย Axel Becke (ปี 1986) ซึ่งมีสมการดังนี้
%
\begin{gather}
    E^{GGA}_{XC} = \int E_{XC} (\rho) * f_{x}(\xi) d^{3}r \\
    \xi = \frac{(\nabla \rho)^{2}}{(2 k_{F} \rho)^{2}}
    \quad , \quad k_{F} = (3\pi^{2}\rho)^{\frac{1}{3}} \\
    f^{B86}_{x}(\xi) = 1 + \frac{a\xi}{1 + b\xi}
\end{gather}
%
\noindent โดยที่ $a = 0.2351$ และ $b = 0.24308$ โดย GGA Functional อันนี้มีโค้เนมที่เรียกกันในวงการคือ B86

หลังจากนั้นอีก 2 ปี (ปี 1988) Becke ก็ได้เสนอ GGA Functional ที่ปรับปรุงมาจาก B86 อีกแล้ว Functional อันใหม่นี้ว่า B88 ซึ่งมีฟังก์ชันของ $f_{x}(\xi)$ ที่ซับซ้อนมากขึ้น ดังนี้ 
%
\begin{equation}
    f^{B88}_{x}(\xi) = 
    1 + \frac{a\xi}{1 + b\sqrt{\xi} arsh (2(6\pi^{2})^{\frac{1}{3}} \sqrt{\xi })}
\end{equation}
%
\noindent แล้วก็ค่าคงที่เปลี่ยนเป็น $a = 0.2743$ และ $b = \frac{9a}{4\pi}$

ส่วน Correlation นั้นจริง ๆ แล้วมี Functional Model หลายอันมาก ๆ ส่วนอันที่ถูกใช้เยอะที่สุดก็คือ PBE (ย่อมาจากนามสกุลของนักเคมีทฤษฎี 3 คนคือ Perdew-Burke-Ernzerhof) ซึ่งก็ถูกนำมาพัฒนาเพิ่มเติมเป็นฟังก์ชันนอล PBE0 โดยผู้อ่านสามารถดูหน้าตาของสมการได้ที่ Wikipedia หรือหนังสือ Electronic Structure ทั่วไป 

ส่วน Approximation อื่น ๆ สำหรับ DFT ในตระกูลนี้นั้นยังมีอีกเยอะเลย มีหลาย ๆ ที่ได้จามาจากการ Fit กับข้อมูลการทดลอง (Empirical Parameters) เช่น B86 หรือ HCTH และ Scaling ของ GGA ก็ดีด้วย ประมาณ $N^{3}$ เท่านั้นเองซึ่งคำนวณได้เร็วกว่า HF มาก ๆ\footnote{ดูรายละเอียดเพิ่มเติมที่ \url{https://tddft.org/programs/libxc/}}

\subsection{Meta-GGA}

Meta-GGA Functional เป็นการยกระดับ GGA ขึ้นมาโดยการใช้ข้อมูลของอนุพันธ์อันดับ 2 ของความหนาแน่น (Second Derivative of the Density) และ/หรือ Kinetic Energy Density เข้ามาช่วยด้วย 

Meta-GGA Functional เช่น TPSS, M06-L, rev-TPSS, แล้วก็ SCAN ให้ผลการคำนวณที่ถูกต้องกว่า GGA 
เพราะว่ามันสามารถ Treat ประเภทของพันธะเคมีในโมเลกุลที่แตกต่างกันได้ เช่น Covalent Bond, Metallic Bond, 
Weak Bond) 

\subsection{Hybrid Functional}

ไอเดียของ Hybrid Functional ก็ตามชื่อเลยคือเป็นการนำ HF มาผสมกับ DFT ซึ่งให้ผลการคำนวณที่ถูกต้องมาก ๆ โดย Functionals ในตำนานยอดฮิตที่ผู้คนต่างก็ยังใช้กันมานานก็คือ B3LYP กับ PBE0 
%
\begin{equation}
    E^{Hybrid}_{XC} = a E^{HF}_{X} + (1 - a) E^{GGA}_{X} + b E^{LDA}_{c}
\end{equation}

\section{ปัญหาของ DFT}

จุดอ่อนหลักของ DFT ก็คือเราไม่มี Systematic Way ในการปรับปรุง XC Potential นั่นคือ Functionals แต่ละอันที่แต่ละเจ้าเสนอมานั้นมันได้มาจากทฤษฎีที่แตกต่างกัน เห็นได้จากจำนวนของ Functionals ที่มีอยู่ในปัจจุบันนี้ที่มีจำนวนมากมายมหาศาล ซึ่งแทบจะเป็นไปไม่ได้เลยที่ XC Functional ในอุดมคติหรือ Universal XC Functional (หรือ Exact Functional) นั้นจะถูกหาเจอ\footnote{Universal Functional เป็น Functional ที่มีสรรพคุณครอบจักรวาลก็คือสามารถนำไปใช้ในการคำนวณกับระบบไหนก็ได้ที่ต้องการ ซึ่งมันคือ Holy Grail 
ของ DFT เลยก็ว่าได้}

ประสิทธิภาพของ Density Functional Approximation (DFA) ที่เป็นตัวกำหนดความถูกต้องของการคำนวณด้วย DFT Model นั้นเรียงตามประเภทของ Approximation Functional ได้ดังนี้
%
\begin{framed}
LDA < GGA < meta-GGA \~ Hybrid Functionals 
\end{framed}
%
\noindent และ 
%
\begin{framed}
GGA < GGA + dispersion Correlation 
\end{framed}

\section{Self-Interaction Error}

ปัญหาคลาสสิคอีกอย่างหนึ่งของ DFT ก็คือ Self-Interaction Error ก็คือว่าใน DFT นั้น อิเล็กตรอนแต่ละตัวนั้นมันจะสัมผัสหรือ Feel ได้ถึงตัวเอง ซึ่งกลายเป็นมันมี Error ที่เกิดจาก Interaction ระหว่างอิเล็กตรอนตัวมันเอง ดังนั้นการแก้ปัญหาดังกล่าวก็คือการตัดหรือลบ Orbital ที่ขึ้นอยู่กับ Self-Interaction ออกไป 

อย่างไรก็ตามได้มีการทำ Correction ให้กับ DFT เช่น วิธี DFT+U methods ซึ่งเป็นการเพิ่มเทอม Hubbard U เข้าไป ซึ่งเป็นการเพิ่ม Correction โดยใช้ Localized Electron Density แต่วิธีนี้ก็มีข้อจำกัดคือความถูกต้องนั้ขึ้นกับ U Parameter นอกจานี้การทำ Correlation แบบนี้นั้นมีความซับซ้อนมากเพราะว่ามันเกี่ยวข้องกับ Orbital แทนที่จะเป็น Density ซึ่งกลายเป็นว่าการที่เราทำแบบนี้ทำให้ทฤษฎีอันนี้มันไม่ใช่ DFT อีกต่อไปแล้ว นอกจากนี้ยังมีปัญหาอื่น ๆ ตามมาอีก เช่น ปัญหาทางด้านการคำนวณเพราะว่าการใช้ Self-Interaction Correlation นั้นมันทำให้การ Converge ของพลังงานนั้นยากขึ้น (ค่าพลังงานที่ได้ออกจากจึงไม่ถูกต้อง) นั่นจึงทำให้วิธี HF นั้นมีข้อได้เปรียบกว่า DFT ในแง่ที่ว่ามันไม่มีปัญหาเรื่อง Self-Interaction จึงเป็นเหตุผลที่ว่าทำไม Hybrid Functional นั้นให้ผลการคำนวณที่ถูกต้องการ GGA Models 
