% LaTeX source for ``Summary of Quantum Chemistry''
% Copyright (c) 2023 รังสิมันต์ เกษแก้ว (Rangsiman Ketkaew).

\chapter{Density Functional Theory}

\section{เกริ่นนำ}

ทฤษฎีทางเคมีควอนตัมแบบดังเดิม (traditional quantum chemical methods) นั้นจะทำการจัดกับ
wavefunction โดยการ represent ด้วยฟังก์ชันแบบต่าง ๆ แล้วก็หาคำตอบของ correlation energy
ออกมาซึ่งมีความสิ้นเปลืองในเชิงการคำนวณสูงมากเพราะว่ามีขั้นตอนต่าง ๆ ที่ซับซ้อนและแทบจะนำไปใช้จริง%
ไม่ได้เลยกับระบบที่มีขนาดใหญ่

Density Functional Theory (DFT) หรือทฤษฎีฟังก์ชันแนลความหนาแน่นนั้นแตกต่างออกไป
ตัว DFT เองนั้นใช้แนวคิดที่ไม่ได้ผูกกับ wavefunction ตามที่เราได้ศึกษากันมาในที่แล้วอีกต่อไป
โดยมีหัวใจหลักก็คือจะมองว่า \enquote{พลังงานของระบบที่สภาวะพื้นนั้นจะขึ้นอยู่กับความหนาแน่น%
    ของอิเล็กตรอนเพียงเดียวเท่านั้น}

\begin{equation}
    \Psi(r_{1}, r_{2}, \dots, r_{N}) \rightarrow \Psi[\rho(r)]
\end{equation}

ซึ่ง wavefunction ในเวอร์ชันของ DFT นั้นง่ายกว่า wavefunction จริง ๆ เยอะเลย
แต่ปัญหาก็คือว่าเราไม่รู้สมการที่จะแก้ wavefunction อันนี้ออกมาได้ ไอเดียอันนี้เริ่มต้นมาจาก
Kohn, Hohenberg และ Sham (ตามปี ค.ศ. ที่ตีพิมพ์เปเปอร์ 1963, 1694) แล้วก็ทำให้วิธี
DFT นั้นได้รางวัลโนเบลสาขาเคมีในปี 1998\footnote{วิธี DFT ได้รางวัลโนเบลเพราะว่าไอเดียของวิธี
    ไม่ใช่ความแม่นยำของวิธีเพราะว่ายังมีวิธีอื่น ๆ อีกเยอะแยะมากมายที่แม่นยำกว่าวิธี DFT แต่เป็นเพราะว่า%
    มีคนนำไปใช้งานเยอะและใช้งานได้จริงกับระบบโมเลกุลตั้งแต่ขนาดเล็กไปจนถึงขนาดใหญ่เพราะว่าใช้เวลา%
    ในการคำนวณเร็วกว่าวิธีอื่น ๆ} แม้ว่าเราจะไม่รู้สมการที่ถูกต้องจริง ๆ ในการแก้ Kohn-Sham wavefunction
แต่ว่าเราก็มีการสร้าง approximation ขึ้นมา โดย Kohn กับ Sham ได้พิสูจน์และพบว่าเพียงแค่เราใช้
wavefunction ที่เป็น single Slater determinant มารวมกับสิ่งที่เรียกว่า Exchange-Correlation
นั้นก็ให้ค่าผลการคำนวณที่ถูกต้องแล้ว

พาร์ทที่น่าสนใจและเป็นปัญหามากที่สุดมาจนถึงปัจจุบันนี้ก็คือ Exchange-Correlation energy เพราะว่า ณ
ปัจจุบันนี้ยังไม่มีใครสามารถหาหน้าตาที่เป็น exact solution ของเทอม XC นี้ได้ ซึ่งฟังก์ชันนอล XC
ที่มีให้เราเลือกใช้กันตามโปรแกรมเคมีคำนวณต่าง ๆ ในปัจจุบันนี้เป็นเพียงแค่ approximation เท่านั้น

\section{Orbital-free DFT}

พลังงานของระบบสามารถเขียนเป็นสมการง่าย ๆ ได้ดังนี้

\begin{equation}
    E[\rho] = T_{s}[\rho] + W[\rho] - \int d^{3} r V(r,R_{I}) \rho(r)
\end{equation}

\noindent $T[\rho]$ คือพลังงานจลน์, $W[\rho]$ คือพลังงานของอันตรกิริยาระหว่างอิเล็กตรอน
(exact electron-electron interaction energy) ซึ่งมีหน้าตาดังนี้

\begin{gather}
    T[\rho] = -\sum_{i} \mel{\psi}{\nabla^{2}_{i}}{\psi} \\
    W[\rho] = \int \frac{\rho(r)\rho(r)'}{|r-r'|} d^{3} r d^{3} r' + E_{XC}(\rho)
\end{gather}

เทอมที่เราไม่รู้นั้นก็คือ $T[\rho]$ และ  $E_{XC}(\rho)$ แต่ว่าเรามี approximation สำหรับทั้งสองเทอมนี้
(อย่างไรก็ตาม approximation ของ $T[\rho]$ ที่เรามีนั้นก็ไม่ค่อยดีเท่าไหร่)

สมการด้านบนนี้เป็นของวิธีที่เรียกว่า density-only DFT ก็คือเป็น DFT ที่ขึ้นอยู่กับ density เท่านั้น
หรือมีชื่อเรียกอีกอย่างในวงการว่า orbital-free DFT ซึ่งให้ผลการคำนวณที่ไม่ถูกต้องเลย

\section{Kohn-Sham DFT}

ลำดับถัดมาคือเป็น DFT scheme อันนึงที่ถูกเสนอโดย Kohn กับ Sham ซึ่งเขาทั้งสองคนนั้นมี assumption
ว่าจริง ๆ แล้วจะมีระบบที่อิเล็กตรอนไม่ขึ้นต่อกัน (non-interaction system) ที่สามารถมี density
ที่ถูกต้องได้ (และระบบที่มี density ถูกต้องนั้นก็จะให้ค่าพลังงานของระบบที่ถูกต้องด้วยเช่นกัน)

สำหรับ non-interaction system นั้น เราจะเริ่มจาก Slater determinant ของ wavefunction
ก่อน ดังนี้

\begin{equation}
    \Psi_{KS} = det[\phi_{1,KS}, \phi_{2,KS}, \dots, \phi_{N,KS}]
\end{equation}

\noindent และมี density คือ

\begin{equation}
    \rho(r) = \sum^{N}_{n=1} \psi_{n,KS}^{\ast}(r) \psi_{n,KS}(r)
\end{equation}

\noindent ซึ่งเราสามารถสร้าง Kohn-Sham equation โดยใช้ Kohn-Sham orbitals ได้ ดังนี้ 
