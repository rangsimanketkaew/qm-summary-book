% LaTeX source for ``Summary of Quantum Chemistry''
% Copyright (c) 2023 รังสิมันต์ เกษแก้ว (Rangsiman Ketkaew).

{
% \pagenumbering{gobble}

\chapter*{\centering สิ่งที่ควรทราบเกี่ยวกับหนังสือเล่มนี้}
\addcontentsline{toc}{chapter}{สิ่งที่ควรทราบเกี่ยวกับหนังสือเล่มนี้}

หนังสือเล่มนี้เป็นการสรุปเคมีควอนตัมเชิงเทคนิคโดยเน้นไปที่การอธิบายภาพรวมของทฤษฎีและวิธีการคำนวณต่าง ๆ ที่เกี่ยวข้องกับโครงสร้างเชิงอิเล็กทรอนิกส์ (Electron structure) ของโมเลกุล มีการสรุปการใช้ฟังก์ชันคลื่นของอิเล็กตรอนและพลังงานของระบบซึ่งคุณสมบัติ(เกือบ)ทั้งหมดของโมเลกุลนั้นสามารถคำนวณได้โดยการใช้วิธีคอมพิวเตอร์ทางเคมีควอนตัม ส่วนการศึกษาพลศาสตร์หรือการเคลื่อนที่ (Dynamics) ของโมเลกุลนั้นเราต้องจำลองระบบให้มีสภาวะและองค์ประกอบทางกายภาพที่สอดคล้องกับการทดลอง เช่น อุณหภูมิหรือความดันที่แน่นอน โดยการจำลองทางคอมพิวเตอร์ดังกล่าวนั้นเรียกว่า Molecular Dynamics ซึ่งไม่ใช่เนื้อหาหลักของหนังสือเล่มนี้ สำหรับผู้สนใจผมแนะนำให้อ่านหนังสือ \href{https://rangsimanketkaew.github.io/archive/algo-sim-mol/}{อัลกอริทึมสำหรับการจำลองทางคอมพิวเตอร์ของระบบโมเลกุล - Algorithms for Computer Simulation of Molecular Systems}

เนื้อหาของหนังสือเล่มนี้เป็นการสรุปทฤษฎีพื้นฐานที่มีความสำคัญในการนำไปต่อยอดในการพัฒนาทฤษฎีที่มีความซับซ้อนมากขึ้น ยิ่งไปกว่านั้น งานวิจัยทางด้าน Electronic Strucuture ในปัจจุบันนั้นถือได้ว่าพัฒนาไปไกลไปไกลและรวดเร็วมาก ๆ มีบทความวิจัยตีพิมพ์เรื่อย ๆ ทุปวัน แม้แต่ตำราต่างประเทศก็อัพเดทไม่ทัน ถ้าหากผู้อ่านต้องการติดตามงานวิจัยใหม่ ๆ ผมแนะนำให้อ่านงานวิจัยตามวารสารชั้นนำด้านเคมีเชิงทฤษฎีคำนวณอย่างสม่ำเสมอ

โชคดีมีชัย โชคชัยมีวัว \quad ขอให้มีความสุขกับการอ่าน ... Enjoy ครับ!

}
