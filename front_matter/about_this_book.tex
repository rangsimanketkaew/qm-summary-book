% LaTeX source for ``Summary of Quantum Chemistry''
% Copyright (c) 2023 รังสิมันต์ เกษแก้ว (Rangsiman Ketkaew).

{
% \pagenumbering{gobble}

\chapter*{\centering สิ่งที่ควรทราบเกี่ยวกับหนังสือเล่มนี้}
\addcontentsline{toc}{chapter}{สิ่งที่ควรทราบเกี่ยวกับหนังสือเล่มนี้}

เอกสารชุดนี้เป็นการสรุปเนื้อของเคมีควอนตัมโดยเน้นไปที่อธิบายภาพรวมของทฤษฎีและวิธีการคำนวณต่าง ๆ 
ของโครงสร้างเชิงอิเล็กทรอนิกส์ (Electron structure) ของโมเลกุล มีการสรุปและอธิบายการใช้%
ฟังก์ชันคลื่นของอิเล็กตรอนและพลังงานของระบบซึ่งคุณสมบัติเกือบทั้งหมดของโมเลกุล (โมเลกุลเดี่ยว ๆ) 
นั้นสามารถคำนวณได้โดยการใช้วิธีทางเคมีควอนตัม แต่ว่าถ้าเป็นระบบที่มีหลาย ๆ โมเลกุลแล้วเราต้องการ%
คำนวณคุณสมบัติที่อุณหภูมิที่แน่นอนนั้นเราจะต้องใช้วิธีพิเศษที่เรียกว่าการจำลองแบบ dynamics 
เข้ามาช่วยซึ่งจะเกี่ยวข้องกับการคำนวณเอนโทรปีของระบบนั่นเอง สำหรับการจำลอง dynamics 
นั้นสามารถทำได้ 2 วิธีคือใช้การจำลอง Molecular Dynamics (MD) กับ Monte Carlo (MC) 
ซึ่งทั้งสองวิธีนี้ผมไม่ได้อธิบายไว้เพราะว่ายังไม่มีเวลาเขียน (แนะนำให้ผู้อ่านไปหาอ่านเพิ่มเติมเองก่อนนะครับ)

เนื้อหาของสรุปเล่มนี้นั้นมีน้อยมาก ๆ เมื่อเทียบกับเนื้อทั้งหมดที่มีในวิชาเคมีควอนตัมและโครงสร้างเชิงอิเล็กทรอนิกส์แล้ว
ยิ่งไปกว่านั้น งานวิจัยทางด้านนนี้ในปัจจุบันถือได้ว่าก้าวไปไกลและเร็วมาก ๆ แม้แต่หนังสือต่างประเทศก็อัพเดทไม่ทัน
ดังนั้นถ้าหากผู้อ่านต้องการติดตามงานวิจัยใหม่ ๆ ผมแนะนำให้ตามอ่านงานวิจัยตามวารสารชั้นนำของเคมีเชิงฟิสิกส์ 
เคมีเชิงทฤษฎีและเคมีคำนวณอย่างต่อเนื่องและสม่ำเสมอครับ

โชคดีมีชัย โชคชัยมีวัว \quad ขอให้มีความสุขกับการอ่าน ... Enjoy ครับ!

}
